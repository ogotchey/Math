\documentclass{article}
\usepackage[utf8]{inputenc}
\usepackage{amsmath}
\usepackage{amssymb}
\usepackage{amsfonts}
\usepackage{float}
\usepackage{amsthm}
\usepackage{graphicx}
\usepackage{fullpage}
\usepackage{color}
\usepackage{tikz-cd}
\usepackage{hyperref}
\usepackage[shortlabels]{enumitem}
\hypersetup{
    colorlinks=true,
    linkcolor=blue,
    urlcolor=red,
    linktoc=all
}
\definecolor{dg}{rgb}{0.0, 0.5, 0.0}
\newcommand{\R}{\mathbb{R}}
\newcommand{\C}{\mathbb{C}}
\newcommand{\N}{\mathbb{N}}
\newcommand{\Q}{\mathbb{Q}}
\newcommand{\Z}{\mathbb{Z}}
\newcommand{\D}{\mathbb{D}}
\newcommand{\HH}{\mathbb{H}}
\newtheorem{theorem}{Theorem}
\newtheorem{example}{Example}
\newtheorem{remark}{Remark}
\newtheorem{proposition}{Proposition}
\newtheorem{lemma}{Lemma}
\newtheorem{corollary}{Corollary}
\newtheorem{definition}{Definition}
\title{Topology Notes - Spring 2022}
\author{JJ Hoo}
\date{January 2022}

\begin{document}

\maketitle
\tableofcontents
\newpage
\section{General Topology}
\subsection{August 30, 2022}
\begin{definition}
Let $X$ be a set. A topology $\tau$ on $X$ is a family of subsets (called open subsets) such that:
\begin{enumerate}
    \item $\emptyset, X\in\tau$
    \item Finite intersections of open subsets are open
    \item Arbitrary unions of open subsets are open
\end{enumerate}
\end{definition}

\begin{definition}
Let $f:X\rightarrow Y$, where $X$ and $Y$ are topological spaces. $f$ is continuous if the preimage of every open set is an open set,
\end{definition}
\begin{proposition}
If $f:X\rightarrow Y$ and $g:Y\rightarrow Z$ are continuous maps, then $g\circ f$ is continuous.
\end{proposition}
\begin{proof}
If $U$ is open in $Z$, then $g^{-1}(U)$ is open in $Y$. Since $f$ is continuous, $f^{-1}(g^{-1}(U))$ is open. Thus, $U$ open implying $(g\circ f)^{-1}(U)$ is open gives us that $g\circ f$ is continuous.
\end{proof}
\noindent \textbf{How should we construct topological spaces?}:
\begin{definition}
A \underline{basis} for a topology is a collection $\mathcal{B}$ of subsets of a set $X$ such that:
\begin{itemize}
    \item $X$is the union of the elements of $\mathcal{B}$
    \item if $B_1,B_2\in\mathcal{B}$, then for every $x\in B_1\cap B_2$, there is a $B_3\in\mathcal{B}$ such that $x\in B_3\subseteq B_1\cap B_2$
\end{itemize}
We define the open sets as arbitrary unions of elements of $\mathcal{B}$.
\end{definition}
\noindent By the first property, $\emptyset, X\in \mathcal{B}$. Arbitrary unions are in the topology by definition. Now, suppose $B_1,B_2\in\mathcal{B}$, then $B_1\cap B_2=\displaystyle\bigcup_{x\in B_1\cap B_2} B_3(x)$, where $B_3$ is as prescribed in the second property. We then get by induction that finite intersections are in the topology.
\begin{example}
Consider $\{(a,b)|a<b,\ a,b\in\R\}$. This forms a basis for the standard topology on $\R$. 
\end{example}
\begin{example}
Consider $\{(a,b)|a<b\ a,b\in\Q\}$. This ALSO forms a basis for the standard topology on $\R$.
\end{example}
\begin{proposition}
$f:X\rightarrow Y$ is continuous if the preimage of every basis element is open.
\end{proposition}
\begin{example}
Consider $C[a,b]$. Consider the function $\phi:C[a,b]\rightarrow \R$ such that $\phi(f)=\int_a^b f(x)\ dx$ is continuous
\end{example}
\begin{example}
Consider $L^p(\R)$, where $p\geq 1$. Consider $\phi:L^p(\R)\rightarrow \R$, where $\phi(f)=\int_{-\infty}^\infty (f(x))^p\ dx$. $\phi$ is continuous.
\end{example}
\begin{definition}
Let $(X,\tau)$ be a topological space. Consider $Y\subseteq X$. We define $\tau_Y = \{U\cap Y|\ U\in \tau\}$ to be the subspace topology. The axioms for a topology are easily checked here.
\end{definition}
\begin{proposition}
Let $(X,\tau)$ be a topological space. Let $\mathcal{B}$ be a family of subsets that satisfy the conditions of a basis. 
If every element of $\mathcal{B}$ is open and if for every open set $U$ and every $x\in U$, there is $B\in\mathcal{B}$ such that $x\in B\subseteq U$, then $B$ is a basis for $\tau$ 
\end{proposition}
\begin{proof}
For every $U\in \tau$, write $U=\displaystyle\bigcup_{x\in U} B(x)$, where $B$ is prescribed as above. 
\end{proof}
\begin{example}
Consider $\{[a,b)|\ a<b, a,b\in\R\}$. This is the lower limit topology. Note that $[1,2)$ is not a union of open intervals, so this is not the standard one. 
$$(a,b)=\displaystyle\bigcup_{n=1}^\infty [a+\frac{1}{n},b)$$
\end{example}
\begin{definition}
$\tau'$ is finer than $\tau$ if $U\in \tau\implies U\in \tau'$. We say here that $\tau$ is coarser than $\tau'$. For instance, 
the lower limit topology is finer than the standard topology.
\end{definition}
\begin{example}
Consider $\mathcal{B}=\{V_n(a)|\ a\in\R^n, n>0\}$.
\end{example}
\begin{example}
Consider $C[a,b]$. The basis consists of the sets:
$$\{g|\ \sup_{x\in[a,b]} |f(x)-g(x)|<r\},\qquad f,g\in C[a,b],\ r>0$$
\end{example}
\begin{example}
Consider $L^p(\R)$, for $p\geq 1$.The basis consists of the sets:
$$\{g|\left(\int_{-\infty}^\infty |f-g|^p\ dx\right)^\frac{1}{p}<r\},\qquad f\in L^p(\R),\ n>0$$
It is of note that these are great examples of Banach Spaces
\end{example}
\subsection{September 1, 2022}
Consider the following diagram for the product topology:
\begin{center}
\begin{tikzcd}
Z \arrow[rd, "f_\alpha"] \arrow[r, "f"] & \displaystyle\prod_{\alpha\in A} X_\alpha \arrow[d, "\pi_\alpha"] \\
                                        & X_\alpha                                                         
\end{tikzcd}
\end{center}
Note that $f_\alpha = \pi_\alpha\circ f$. $f$ is continuous iff $f_\alpha$ is continuous for all $\alpha$. We want each of the projections $\pi_\alpha$ to be continuous.\\
\\
Note that $\pi_\alpha^{-1}(U_\alpha)$ is open for every open $U_\alpha\subseteq X_\alpha$, because:
$$\pi^{-1}(U_\alpha)=U_\alpha\times \displaystyle\prod_{\beta\neq \alpha}X_\beta$$
Hence, a basis for this topology are sets of the form:
$$U_{\alpha_1}\times U_{\alpha_2}\times\cdots\times U_{\alpha_n}\times\displaystyle\prod_{\beta\neq \alpha_i}X_\beta\qquad U_{\alpha_i}\in X_{\alpha_i}\text{ open}$$
\begin{proposition}
$f$ is continuous if and only if $f_\alpha$ is continuous for every $\alpha$.
\end{proposition}
\begin{proof}
The forward direction is trivial. Since $\pi_\alpha$ is continuous, $f_\alpha=f\circ\pi_\alpha$ is a composition of continuous maps, and is thus continuous. Now, let $f_\alpha$ be continuous for every $\alpha$. We only need to check that the preimage of a basis element is an open set.
\begin{align*}
    f^{-1}\left(U_{\alpha_1}\times U_{\alpha_2}\times\cdots\times U_{\alpha_n}\times \displaystyle\prod_{\beta\neq \alpha_i}X_\beta\right) &= f^{-1}\left(\bigcap_{i=1}^n\left(U_{\alpha_i}\times\displaystyle\prod_{\beta\neq \alpha_i}X_\beta\right)\right)\\
    &=\bigcap_{i=1}^n f^{-1}\left(U_{\alpha_i}\times\displaystyle\prod_{\beta\neq \alpha_i}X_\beta\right)\\
    &=\bigcap_{i=1}^nf^{-1}_{\alpha_i}(U_{\alpha_i})
\end{align*}
This is open because the preimages of $U_\alpha$ by $f_\alpha$ are open, and finite intersections of open sets are open
\end{proof}
\begin{example}
Consider $C[a,b]\subseteq\displaystyle\prod_{x\in[a,b]}\R$. This topology has as basis sets of the form:
$$V_{f,x_1,\cdots, x_n,\epsilon}=\{g:\ |f(x_i)-g(x_i)|<\epsilon,\ i\in\N\}$$
This is called the $\text{weak}^*$ topology on $C[a,b]$.
\end{example}
\begin{example}
Consider $\R\times\R\rightarrow\R$. we can define this to have multiplication, addition, or division defined on $\R\times \R\setminus\{0\}$.
\end{example}
\begin{proposition}
If $X$ is a topological space and $f,g:X\rightarrow \R$ (with maybe $g:X\rightarrow \R\setminus\{0\}$) are continuous functions, then $f+g, f\cdot g, \frac{f}{g}$ are continuous.
\end{proposition}
\begin{proof}
These are compositions of continuous functions. We consider:
$$X\xrightarrow{(f,g)}\rightarrow \R\times\R\xrightarrow{+}\R$$
The same happens for multiplication and division.
\end{proof}
\noindent\textbf{Disjoint Unions of Topological Spaces:}\\
\\
Consider $X_\alpha, \alpha\in A$, $X_\alpha\cap X_\beta=\emptyset$. We then denote the disjoint union as $\displaystyle\bigsqcup_{\alpha\in A}X_\alpha$. Consider the following diagram: 
\begin{center}
    % https://tikzcd.yichuanshen.de/#N4Igdg9gJgpgziAXAbVABwnAlgFyxMJZABgBpiBdUkANwEMAbAVxiRAB12os40G6AnnBwCGMTgCMsAczgBHAMZM0AfWCdGaABZ1OWMAAIAggF8AGio0NtdECdLpMufIRQBGclVqMWbAJp2DiAY2HgERGRuXvTMrIggFlY2dl4wUNLwRKAAZgBOEAC2SABM1DgQSB7esWzZluyaOiDU-BIwDAAKTmGuILkyWjiBOflFiKUg5Uhkk3RYDGxaEBAA1sMgeYXTZRWIVTG+8dkpJkA
\begin{tikzcd}
\displaystyle\bigsqcup_{\alpha\in A}X_\alpha \arrow[r, "f"] & Y \\
X_\alpha \arrow[ru, "f_\alpha"'] \arrow[u, hook]            &  
\end{tikzcd}
\end{center}
Similarly as before, $f$ is continuous if and only if $f_\alpha$ is continuous for all $\alpha$. A topology consists of the sets that are unions of $U_\alpha$ where $U_\alpha$ are open in $X_\alpha$. If the $X_\alpha$'s are not disjoint, then we might run into trouble in the overlap!
\\
\\
\noindent\textbf{Quotient Spaces}
Consider a topological space $X$. Take $f:X\rightarrow Y$ to be a surjective function. THe open sets of $Y$ are of the form $f(U)$ where $U$ is open in $X$.\\
\\
Another perspective: Consider a topological space $X$. Take an equivalence relation on $X$. Denote by $\hat{x}$ the equivalence class of $x$. Let:
$$Y=\{\hat{x}|\ x\in X\}$$
Now, we consider the function $f:X\rightarrow Y$ where $f(x)=\hat{x}$., and we use the notation $X/\sim$ to represent this quotient space. Now, consider the following diagram:
\begin{center}
% https://tikzcd.yichuanshen.de/#N4Igdg9gJgpgziAXAbVABwnAlgFyxMJZABgBoBGAXVJADcBDAGwFcYkQBNEAX1PU1z5CKMsWp0mrdgA0efEBmx4CRcqTE0GLNohAAtHuJhQA5vCKgAZgCcIAWyQAmGjghIyE7exMgajegBGMIwACgLKwiCMMJY4viBwABZYse68VrYOiGogru6akjoglvH+QaHhQuzWWCaJcenFmU4ubtkFXromADrdAMZY1n0ABCV+gcFhSlW6NXUNlNxAA
\begin{tikzcd}
X \arrow[d, "f"'] & Z \arrow[ld, "g"] \arrow[l, "g\circ f"'] \\
Y                 &                                         
\end{tikzcd}
\end{center}
We want $g$ to be continuous if and only if $g\circ f$ is continuous, and this is the reason for our construction of the quotient space.
\begin{proposition}
The above is a topology on $Y$.
\end{proposition}
\begin{proof}
\begin{enumerate}[(a)]
    \item $Y=f(x)$ is open, and $\emptyset=f(\emptyset)$ is open
    \item $f(U_1\cap U_2\cap\cdots\cap U_n)$ is a finite intersection of $f(U_i)$'s, so this is open
    \item If $f(U_\alpha)$ is open, for all $\alpha$, then it's easy to see that the images commute with the union, and so arbitrary unions are open.
\end{enumerate}
\end{proof}
\begin{example}
Take $f:\R\rightarrow\C$. This may not be a surjection, but it is certainly a surjection onto the image by construction. Let $f(x)=e^{2\pi ix}$. The image is $\{z:\ |z|=1\}=: S^1$ (the 1-dimensional sphere). We can induce a topology on $S^1$ by using the standard topology on $\R$, or by inducing from the topology on $\C$. The induced topology is the same as the quotient topology. This can be resolved by thinking of the standard convention where $S^1=\R/\Z$.
\end{example}
\begin{example}
Consider $S^2$, the sphere in $\R^3$. Consider the inclusion of $S^2\subseteq \R^3$, and consider the induced topology on $S^2$.  We can consider two points on the sphere to be equivalent as follows:
$$(x,y,z)\sim (x',y',z')\iff (x=x',y=y',z=z')\vee (x=-x',y=-y',z=-z')$$
Let $\R P^2 = S^2/\sim$, and consider the quotient topology. This space is essentially a cap being put on top of a M\"obius band. By definition, this is the two-dimensional projective plane. Another way to construct $\R P^2$:\\
\\
consider $\R^3\setminus\{(0,0,0)\}\subseteq \R^3$ using the standard topology. We take the following equivlence:
$$(x,y,z)\sim (x',y',z')\iff \exists \lambda \neq 0:\ x=\lambda x', y=\lambda y', z=\lambda z'$$
Then, $(\R^3\setminus\{(0,0,0)\})/\sim$ is exactly $\R P^2$. We can then generalize this construction to $\R P^n$ using the same procedure, but we can also have $\C P^n$ to be equivalence classes of points in $\C^{n+1}\setminus\{(0,0,\cdots,0)\}$, with the same equivalence as before, except now $\lambda\in \C$. For instance $\C P^1$ is the Riemann sphere. 
\end{example}
\begin{example}
Consider the unit square with corners $(0,0), (0,1), (1,0), (1,1)$. We induce an equivalence relation by $(x,1)\sim (x,0)$ for every $x$, and $(1,y)\sim (0,y)$ for every $y$. This generates a torus. This comes from the induced topology from the inclusion into $\R^2$, so the torus is the quotient topology $\R^2/\Z^2$. The torus can also be described as $s^1\times S^1$.
\end{example}
\begin{definition}
$f:X\rightarrow Y$ is a \underline{homeomorphism} if $f$ is continuous, invertible, and $f^{-1}$ is continuous. 
\end{definition}
\begin{example}
$([0,1]\times [0,1])/\sim$ is homeomorphic to $S^1\times S^1$, where $\sim$ is the equivalence relation defined in Example 11.
\end{example}
\begin{example}
Consider $[1,2],[3,4]\subseteq \R$ with the induced topology. Then, take the disjoint union $[1,2]\sqcup [3,4]$. Then, we take the quotient where $1\sim 2 \sim 3\sim 4$. This gives us a figure 8, where the circles touch at $\{1,2,3,4\}$. 
\end{example}
\begin{example}
Let $X$ be a topological space. Consider $[-1,1]$ with the induced topology. Consider $X\times [-1,1]$ with the product topology. We then consider $(x,1)\sim (y,1)$ and $(x,-1)\sim (y,-1)$ for every $x$ and $y$, and take the quotient topology. This then gives a two-sided cone with a circular middle, called the suspension $\Sigma X$. 
If we consider $S^n$, then $\Sigma S^n\cong S^{n+1}$.
\end{example}
\newpage
\subsection{September 6, 2022}
Thus far, as a recap, we have covered the following constructions:
\begin{itemize}
    \item Induced Topology
    \item Product Topology
    \item Disjoint Sum
    \item Quotients
\end{itemize}
We now discuss metric spaces.
\begin{definition}
A distance on a set $X$ is a function $d:X\times X\rightarrow [0,\infty)$ such that:
\begin{enumerate}[(a)]
    \item $d(x,y)=0\iff x=y$
    \item $d(x,y)=d(y,x)$
    \item $d(x,z)\leq d(x,y)+d(y,z)$
\end{enumerate}
\end{definition}
A metric is then another word we use instead of ``distance". $(X,d)$ is thus termed a metric space.
\begin{example}
$(\R^n,d)$, where $d(\vec{x},\vec{y})=\sqrt{\displaystyle\sum_{i=1}^n(x_i-y_i)^2}$ 
\end{example}
\begin{example}
$(\R^n,d)$, where $d(\vec{x},\vec{y})=\displaystyle\max_{1\leq i \leq n}{|x_i-y_i|}$
\end{example}
\begin{example}
$([0,1]^A,d)$, where $(x_\alpha)_\beta,(y_\alpha)_\beta\in\displaystyle\prod_{\beta\in B}[0,1]$, and:
$$d((x_\alpha)_\alpha,(y_\alpha)_\alpha)=\sup_\alpha{|x_\alpha-y_\alpha|}$$
\end{example}
A basis for the topology induced by the metric consists of the balls (for $x\in X$ and $\epsilon >0$):
$$B(x,\epsilon)=\{y|d(x,y)<\epsilon\}$$
We now prove that these open balls form a basis.
\begin{proof}
First, we must check that these cover the entire space. However, it is clear that $X=\displaystyle\bigcup_{x\in X}B(x,\epsilon)$. For another condition, we see that if $B_1=B(x_1,\epsilon_1)$ and $B_2=B(x_2,\epsilon_2)$ are elements of the basis, and if $x\in B_1\cap B_2$, we have to show that there is a $B_3=B(x_3,\epsilon_3)$ such that $x\in B_3\subseteq B_1\cap B_2$. We can construct this geometrically such that we pick $x=x_3$ and $\epsilon_3 <\min{(\epsilon_1-d(x,x_1),\epsilon_2-d(x,x_2))}$. Then, if $y\in B_3$, we have that:
\begin{align*}
    d(y,x_1)&<d(y,x)+d(x,x_1)\\
    &<\epsilon_3 + d(x,x_1)\\
    &<\epsilon_1-d(x,x_1)+d(x,x_1)\\
    &=\epsilon_1
\end{align*}
Thus, $y\in B_1$, and by the exact same argument, $y\in B_2$, so we have indeed proven that this does form a basis.
\end{proof}
We now note that open sets are unions of balls.
\begin{example}
On $\R^n$, the standard metric and the max metric from Examples 17 and 18 induce the same topology.
\end{example}
\begin{proposition}
If two metrics have the property that for every ball of the first and every element of it, there is a ball of the second topology centered at this element and included in the first ball, then the second topology is finer than the first.
$$x\in B'(x,\epsilon')\subseteq B(y,\epsilon),\qquad \forall x\in B(y,\epsilon)$$
\end{proposition}
\noindent In Example 20, a disc can be written as a union of squares, and vice versa.
\begin{example}
If $(X,d)$ is a metric space, define:
$$\overline{d}(x,y)=\min{(d(x,y),1)}$$
Then, $\overline{d}$ is a metric that induces the same topology. \\
\\
We first check that $\overline{d}$ is a metric by ensuring that it satisfies the triangle inequality. If $d(x,y),d(y,z),d(x,z)<1$, then:
\begin{align*}
    \overline{d}(x,z)&=d(x,z)\\
    &\leq d(x,y) + d(y,z)\\
    &=\overline{d}(x,y)+\overline{d}(y,z)
\end{align*}
We have a similar case where $d(x,y)$ or $d(y,z)>1$, because we then have $1\leq \overline{d}(x,y)+\overline{d}(y,z)$ holds. Finally, if $d(x,y),d(y,z)<1$, and $d(x,z)>1$, then:
$$\overline{d}(x,z)\leq d(x,z)<d(x,y) + d(y,z)=\overline{d}(x,y)+\overline{d}(y,z)$$
\end{example}
\begin{theorem}
If $(X,d)$, $(Y,d')$ are metric spaces, then $f:X\rightarrow Y$ is continuous at $x\in X$ if $\forall \epsilon >0$, $\exists \delta >0$ such that $f(B(x,\delta))\subseteq B(f(x),\epsilon)$.
\end{theorem}
\begin{definition}
$f:X\rightarrow Y$ is called uniformly continuous if $\forall \epsilon >0$, $\exists \delta >0$ such that if $d(x,y)<\delta$, then $d'(f(X),f(y))<\epsilon$.
\end{definition}
\begin{definition}
A sequence $(x_n)_{n\geq 1}$ in the metric space $(X,d)$ is called Cauchy if $\forall \epsilon >0$, there exists an $N\in\N$ such that for all $m,n\geq N$, $d(x_m,x_n)<\epsilon$.
\end{definition}
\begin{definition}
A metric space is called \underline{complete} if every Cauchy sequence is convergent.
\end{definition}
\begin{definition}
A \underline{normed vector space} $V$ is a vector space endowed with a function
$\|\cdot \|: V\rightarrow [0,\infty)$ such that:
\begin{enumerate}[(a)]
    \item $\|x\|=0 \iff x=0$
    \item $\|\lambda x\|=|\lambda| \|x\|,\qquad \lambda\in \R\vee \C$
    \item $\|x+y\|\leq \|x\|+\|y\|$
\end{enumerate}
We define $d(x,y)=\|x-y\|$.
\end{definition}
\begin{definition}
A \underline{Banach space} is a complete normed vector space.
\end{definition}
\begin{example}
$C[0,1]$, $L^p$ spaces, Sobolev spaces, Hardy spaces, Bergman spaces.
\end{example}
\newpage
\noindent \textbf{Manifolds:}
\begin{definition}[Veblen]
A manifold $M$ is a Hausdorff topological space endowed with an onto function $$f:\displaystyle\bigsqcup_{\alpha\in A} U_\alpha\rightarrow M$$ such that for every $\alpha$, $U_\alpha$ is an open set in some $\R^n$, where $n$ is the same for all $\alpha$, and $f|U_\alpha$ is a homeomorphism onto the image for each $\alpha$. 
\end{definition}
We can define charts here as something of the form:
$$f_\alpha = f|U_\alpha: U_\alpha\rightarrow f(U_\alpha)\subseteq M$$
A collection of charts is then an atlas. Also, we get compositions where:
$$f_\beta^{-1}\circ f_\alpha: U_\alpha\rightarrow U_\beta$$
\begin{definition}
We call a topological space \underline{Hausdorff} if for every $x\neq y$, there are open sets $U,V$ such that $x\in U$ and $y\in V$, and $U\cap V=\emptyset$.
\end{definition}
\begin{itemize}
    \item If $f_\beta^{-1}\circ f_\alpha$ is smooth, then the manifold is called smooth. 
    \item If $\R^n$ is replaced by $\C^n$ and $f_\beta^{-1}\circ f_\alpha$ is holomorphic or analytic then $M$ is called complex.
\end{itemize}
\begin{example}
The circle $S^1=\{z:\ |z|=1\}$. Define $f:\R\rightarrow \C$ such that $f(t)=e^{2\pi it}$. Let $U_1=(-\pi, \pi)$ and $U_2=(0,2\pi)$. Now, we consider a function $g: U_1\sqcup U_2\rightarrow \C$, where $g(t)=e^{2\pi it}$. We can consider $f(U_1)\cap f(U_2)$ as the overlap, and so $f_1^{-1}(f(U_1)\cap f(U_2))=(=\pi,0)\cup (0,\pi)$, and $f_2^{-1}(f(U_1)\cap f(U_2))=(\pi,2\pi)\cup (0,\pi)$.\\
\\
Then, $f_2^{-1}\circ f_1 :(-\pi,0)\cup (0,\pi)\rightarrow (\pi,2\pi)\cup (0,\pi)$ is as follows:
\begin{equation*}
    f_2^{-1}\circ f_1(t)=\begin{cases}
    t &,t\in(0,\pi)\\
    t+2\pi &,t\in(-\pi,0)
    \end{cases}
\end{equation*}
\end{example}
\begin{theorem}
If $M_1,M_2$ are manifolds of dimensions $n_1, n_2$, then $M_1\times M_2$ is a manifold of dimension $n_1+n_2$.
\end{theorem}
 Consider:
 $$f:\bigsqcup U_\alpha\rightarrow M_1$$
 $$g:\bigsqcup V_\beta\rightarrow M_2$$
 $$f\times g:\bigsqcup U_\alpha\times V_\beta\rightarrow M_1\times M_2$$
 $$(f\times g)_{\alpha\beta}=(f_\alpha,g_\beta)$$
 \begin{example}
 $(S^1)^{\times n}$ is a manifold. In the $n=2$ case, we get $(-\pi, \pi)^2\sqcup (-\pi, \pi)\times (0,2\pi)\sqcup (0,2\pi)\times (-\pi,\pi)\sqcup (0,2\pi)^2$ as our domain, where $f(t,s)=(e^{2\pi it},e^{2\pi is})$, and this gives us the torus.
 \end{example}
 \newpage
 \subsection{September 8, 2022}
 \begin{example}
 Consider $\R P^2=\{\hat{x}:\ x\in\R^3\setminus \{0\},x\sim y\iff x=\lambda y, \lambda\neq 0\}$. $\R P^2$ is a manifold, with the following three charts:
 $$U_1=U_2=U_3=\R^2$$
 $$f: \displaystyle\bigsqcup_{i=1}^3 U_i\rightarrow \R P^2$$
 $$f_1(x_1,x_2)=[1:x_1:x_2]$$
 $$f_2(x_1,x_2)=[x_1:1:x_2]$$
 $$f_3(x_1,x_2)=[x_1:x_2:1]$$
 These maps are one-to-one, and onto. We note that $f_2^{-1}\circ f_1: \R^2\rightarrow \R^2$ is the map $(x_1,x_2)\xrightarrow{f_1} [1:x_1:x_2]=\left[\frac{1}{x_1}:1:\frac{x_2}{x_1}\right]\xrightarrow{f_2^{-1}}\left(\frac{1}{x_1},\frac{x_2}{x_1}\right)$
 \end{example}
 \begin{example}
 Consider $\C P^1$. We have that $U_1=U_2=\C$. We define maps as:
 $$f_1:U_1\rightarrow \C P^1$$
 $$f_1(z)=[1:z]$$
 $$f_2: U_2\rightarrow \C P^1$$
 $$f_2(z)\rightarrow [z:1]$$
 $$f_2^{-1}\circ f_1:z\rightarrow [1:z]\rightarrow \left[\frac{1}{z}:1\right]\rightarrow \frac{1}{z}$$
 \end{example}
 \begin{definition}[Poincar\'e]
 A smooth manifold is a subspace of $\R^n$ that is locally the graph of a smooth function.
 \end{definition}
 
 \begin{example}
 Consider $S^2$. Consider the graph $f(x_1,x_2)=\pm\sqrt{1-x_1^2-x_2^2}$, and similar considerations for the two other orientations. These graphs act as charts from the plane to the sphere.  
 \end{example}
 \noindent Now, we return to closed sets!
 \begin{example}
 $[a,b], [a,\infty)$
 \end{example}
 \begin{example}
 $$C=[0,1]\setminus\displaystyle\bigcup_{n=1}^\infty \bigcup_{k=0}^{3^{n-1}-1}\left(\frac{3k+1}{3^n},\frac{3k+2}{3^n}\right)$$
 This set consists of all numbers in $[0,1]$ that admit a ternary representation with only 0's and 2's. Consider a function:
 $$f: C\rightarrow [0,1]^2$$
 $$f(0.a_1a_2a_3\cdots) = (0.\frac{a_1}{2}\frac{a_3}{2}\cdots, 0.\frac{a_2}{2}\frac{a_4}{2}\cdots)$$
 This function is then continuous and onto from $C$ to $[0,1]^2$. By Tietze's Extension Theorem, we get an extention $\tilde{f}:[0,1]\rightarrow [0,1]^2$ which is continuous and onto. This was induced by G. Peano. 
 \end{example}
 \begin{example}
 Sierpi\'nski Triangle. 
 \end{example}
 \begin{example}
 In the discrete topology, every set is clopen
 \end{example}
 \begin{example}
 $\Q\subseteq \R$, $(a,b)$ are clopen in $\Q$.
 \end{example}
 \begin{example}
 In $\R^n$, consider $\overline{B}(x,\epsilon)=\{y:\ d(x,y)\leq \epsilon\}$ as the closed balls.
 \end{example}
 \begin{example}
 The Zariski topology, where $f(z_1,z_2,\cdots, z_n)\in \C[z_1,\cdots,z_n]$, and:
 $$V(f)=\{(z_1,\cdots, z_n)\in \C^n:\ f(z_1,\cdots, z_n)=0\}$$
 These are called algebraic sets, or varieties if $f$ is irreducible, and these are the closed sets of the Zariski topology. Unions of algebraic sets are the closed sets of the Zariski topology.
 \end{example}
 \begin{proposition}[MAYBE ON EXAM 1]
 \begin{enumerate}[(1)]
     \item If $Y$ is a subspace of $X$, then $A\subseteq Y$ is closed if and only if $A=B\cap Y$ where $B\subseteq X$ is closed.
     \item Let $A\subseteq Y\subseteq X$. If $A$ is closed in $Y$ and $Y$ is closed in $X$, then $A$ is closed in $X$.
     \item If $A$ is closed in $X$ and $B$ is closed in $Y$, then $A\times B$ is closed in $X\times Y$.
     \item If $A_\alpha$ is closed in $X_\alpha$ for $\alpha\in I$, then $\displaystyle\prod_{\alpha\in I}A_\alpha$ is closed in $\displaystyle\prod_{\alpha\in I}X_\alpha$ with the product topology.
 \end{enumerate}
 \end{proposition}
 \begin{proof}
 \begin{enumerate}[(1)]
     \item $Y\setminus A$ is open. So there is a $U$ such that $Y\setminus A = Y\cap U$. Let $B=X\setminus U$. Then $$A=Y\setminus (Y\setminus A) = Y\setminus (Y\cap U) = Y\setminus U=Y\cap (X\setminus U)=B\cap Y$$
     Now, if $A=B\cap Y$, then $Y\setminus A=(X\setminus B)\cap Y$.
     \item $A$ closed in $Y$ and $Y$ closed in $X$ gives us that $X\setminus Y=U$ is open in $X$. Then, $A=Y\cap (X\setminus V)$. Thus $X\setminus A = V\cup U$, so since this is a union of two open sets, it is open, and thus $A$ is closed in $X$.
     \item $(X\times Y)\setminus (A\times B)= (X\times (Y\setminus B))\cup ((X\setminus A)\times Y)$. These sets are all open, so $A\times B$ is closed.
     \item Induction! Do the same thing as above as unions.
     $$\displaystyle\left(\prod_{\alpha\in I}X_\alpha\right) \setminus\left(\prod_{\alpha\in I}A_\alpha\right)=\bigcup_{\alpha\in I}\left(\left(\prod_{\beta\neq \alpha}X_\beta\right)\times (X_\alpha\setminus A_\alpha) \right)$$
 \end{enumerate}
 \end{proof}
 \begin{proposition}
 Let $X,Y$ be topological spaces. Then, $f:X\rightarrow Y$ is continuous if and only if the preimage of every closed set is closed.
 \end{proposition}
 \begin{proof}
 $$f^{-1}(Y\setminus A)=X\setminus f^{-1}(A)$$
 \end{proof}
 \begin{definition}
 The \underline{closure} of a set $S$ is the smallest closed set containing $S$, denoted $\overline{S}$.
 \end{definition}
 \begin{definition}
 The \underline{interior} of a set $S$ is the largest open set contained in $S$, denoted $Int(S)$.
 \end{definition}
 \begin{example}
 $\overline{\Q}-\R$. $Int(\Q)=\emptyset$
 \end{example}
 \begin{example}
 $B(x,r)=\{y:\ d(x,y)<r\}\implies \overline{B(x,r)}=\{y:\ d(x,y)\leq r\}$.
 \end{example}
 \begin{lemma}
 Let $X$ be a topological space, and $A\subseteq X$. Then, $\overline{X\setminus A}=X\setminus Int(A)$
 \end{lemma}
 \begin{proof}
 $$X\setminus A\subseteq X\setminus Int(A)\implies \overline{X\setminus A}\subseteq \overline{X\setminus Int(A)}=X\setminus Int(A)$$
 Now, let $x\in X\setminus (\overline{X\setminus A})$. Then there is an open set $U$ such that $x\in U$ and $U\cap (\overline{X\setminus A})=\emptyset$. Thus, $U\cap (X\setminus A)=\emptyset$, so $U\subseteq A$. Therefore, $x\in Int(A)$.  Hence, $x\not\in X\setminus Int(A)$, so $X\setminus Int(A)\subseteq \overline{X\setminus A}$, and we have the double inclusion as desired.
 \end{proof}
 \newpage
 \begin{theorem}
 Let $A\subseteq X$. Then, $x\in\overline{A}$ if and only if for every open set $U$ such that $x\in U$, we have $U\cap A\neq \emptyset$.
 \end{theorem}
 \begin{proof}
 First, we wish to show that If $x\in\overline{A}$ and $U$ is open where $x\in U$, then $U\cap A\neq\emptyset$. If, for the sake of contradiciotn, $U\cap A=\emptyset$, then $U\subseteq Int(X\setminus A)$, so $X\setminus Int(X\setminus A)=\overline{X\setminus (X\setminus A)}=\overline{A}$, which does not contain $x$, and so we get a contradiction.\\
 \\
 Conversely, if $x$ has the property that every open set $U$ containing $x$ intersects $A$, then $x\not\in Int(X\setminus A)$. So $x\in X\setminus Int(X\setminus A)=\overline{X\setminus(X\setminus A)}=\overline{A}$. Thus, we are done.
 \end{proof}
 \begin{proposition}[Midterm maybe]
 \begin{enumerate}[1)]
     \item Let $Y\subseteq X$, with the subspace topology. If $A\subseteq Y$, then let us denote $\overline{A}_X$ as the closure of $A$ in $X$. Then, the closure of $A$ in $Y$ is $\overline{A}_X\cap Y$.
     \item Taking the same conditions as above, if $Y$ is closed in $X$, then the closure of $A$ in $X$ and $Y$ is the same.
     \item $\displaystyle \prod_{\alpha\in I}\overline{A}_\alpha=\overline{\prod_{\alpha\in I}A_\alpha}$
 \end{enumerate}
 \end{proposition}
 \begin{proof}
 \textbf{1):}\\
 \\
 Let $x\in\overline{A}_X\cap Y$. Then, for every open set $U$ in $X$ containing $x$, $U\cap A\neq \emptyset$. So $(U\cap Y)\cup A\neq \emptyset$. But $U\cap Y$ with $U$ open in $X$ are all open subsets of $Y$. By Theorem 3, $x$ is in the closure of $A$ in $Y$. Thus, $\overline{A}_X\cap Y\subseteq \overline{A}_Y$.\\
 \\
 Conversely, $\overline{A}_X\cap Y$ is closed and contains $A$, so the closure of $A$ in $Y$ is contained in $\overline{A}_X\cap Y$.\\
 \\
 \textbf{2):}\\
 \\
 if $Y$ is closed, then $\overline{A}_X\subseteq \overline{Y}=Y$. So $\overline{A}_X\cap Y = \overline{A}_X$.\\
 \\
 \textbf{3):}\\
 \\
 Let $x=(x_\alpha)_\alpha\in \displaystyle\prod_{\alpha\in I}\overline{A}_\alpha$. Let $\displaystyle\prod_{j=1}^nU_{\alpha_j}\times \prod_{\beta\neq \alpha_j}X_\beta$, or $\displaystyle\prod_{\alpha\in I}U_\alpha$ such that $x_\alpha\in U_\alpha$. Then, by Theorem 3, $U_\alpha\cap A_\alpha\neq \emptyset$, for all $\alpha$, and $X_\beta\cap A_\beta\neq\emptyset$, for all $\beta\neq\alpha$. It follows that in either case, the open set intersects $\displaystyle\prod A_\alpha$. Thus, $\displaystyle\prod \overline{A}_\alpha\subseteq \overline{\prod A_\alpha}$. Conversely, let $x\in \overline{\displaystyle\prod A_\alpha}$. Then, every open set containing $x$ intersects $\displaystyle\prod A_\alpha$. Choose $U$ of either form as expressed in the previous direction, and we get coordinate-wise nonempty intersections. Vary $U_\alpha$'s to make them be any open set that contains $x_\alpha$. You then obtain the reverse inclusion, and thus we are done.      
 \end{proof}
 \noindent This result does not hold for interiors as seen in the following examples:
 \begin{example}
 $\Q\subseteq \R$. $Int*\Q)\neq \emptyset$,  but the interior of $\Q$ in $\Q$ is $\Q$. 
 \end{example}
 However, the third case of the proposition works for interiors in the box topology but not the product topology. 
 \begin{example}
 $[0,1]\subseteq \R$, $Int([0,1])=(0,1)$. However:
 $$\displaystyle\underbrace{\prod_{i=1}^\infty (0,1)}_{\text{not an open set}}\subseteq \prod_{i=1}^\infty\R$$
 \end{example}
 \begin{proposition}
 Let $X,Y$ be topological spaces. $f:X\rightarrow Y$ is continuous if and only if for every subset $A\subseteq X$, we have $f(\overline{A})\subseteq \overline{f(A)}$
 \end{proposition}
 \begin{proof}
 Assume $f$ is continuous. Then $f^{-1}(Y\setminus \overline{f(A)})$ is open. This is the complement of $f^{-1}(\overline{f(A)})$, so this set is open. Then $A\subseteq f^{-1}(\overline{f(A)})$, so $\overline{A}\subseteq f^{-1}(\overline{f(A)})\implies f(\overline{A}\subseteq \overline{f(A)}$.\\
 \\
 Conversely, let $B\subseteq Y$ be open. we want to show $f^{-1}(B)$ is open. Let $A=X\setminus f^{-1}(B).$ Then:
 $$f(\overline{A})\subseteq \overline{f(A)}=\overline{f(X\setminus f^{-1}(B))}=\overline{Y\setminus B} = Y\setminus B$$
 Thus, $f(\overline{A}\subseteq Y\setminus B$, but $A=X\setminus f^{-1}(B)$, so $A=\overline{A}$, and we are done
 \end{proof}
 \begin{definition}
 $x$ is a \underline{limit point} for $A$ if for every $U$ open such that $x\in U$, there is $x'\neq x$ such that $x'\in A\cap U$. We denote $A'$ to be the set of limit points of $A$
 \end{definition}
 \begin{example}
 $$A=\{\frac{1}{n}:\ n\in\N\}$$
 $A$ has only 1 limit point, namely 0. 
 \end{example}
 \begin{proposition}
 $$\overline{A}=A\cup A'$$
 \end{proposition}
 \begin{corollary}
 A subset of a topological space is closed if and only if it contains all of its limit points.
 \end{corollary}
 \begin{definition}
 In an arbitrary topological space, one says that a sequence is convergent to $x\in X$ if for every neighbourhood $V$ of $x$, there is an $N$ such that for all $n\geq N$, $x_n\in V$
 \end{definition}
 \noindent Note: IN the Zariski topology, a sequence with no constant subsequence converges to every point in the space. The definition is fine in metric spaces. In every metric space, the limit is unique.
 \begin{proposition}
 In metric spaces, the limit is unique. 
 \end{proposition}
 \begin{proof}
 If $x_1$, $x_2$ are both limits of a sequence, consider the open ball with radius $\frac{d(x_1,x_2)}{2}$. All but finitely many terms lie in each of the disjoint balls. 
 \end{proof}
 \begin{lemma}[Sequence Lemma]
 Let $X$ be a metric space.
 \begin{enumerate}[(a)]
     \item $x\in\overline{A}$ if and only if there is a sequence in $A$ converging to $x$.
     \item $x\in A'$ if and only if there is a sequence of points in $A$ that does not eventually become constant that converges to $x$. 
 \end{enumerate}
 \end{lemma}
 \begin{proof}
 Recall that $\overline{A}= A\cup A'$. If $x\in A$, then $x_n=x$ for all $n$ converges to $x$. Thus, it suffices to prove part $(b)$.\\
 \\
 Let $x\in A'$. For $n\in\N$, consider $B(x,\frac{1}{n})$. Since $x\in A'$, there is $x_n\in A\cap B(x,\frac{1}{n}$ such that $x_n\neq x$. Now, let $\epsilon >0$, and choose $K(\epsilon)$ so that $\frac{1}{K(\epsilon)}<\epsilon$. Then, for $n\geq K(\epsilon)$, $x_n\in B(x,\frac{1}{n})\subseteq B(x,\frac{1}{K(\epsilon)}\subseteq B(x,\epsilon)$. Convsersely, if $x_n\rightarrow x$, $x_n\in A$, $x_n$ not eventually constant. We consider an open set $U$ containing $x$. Let $B(x,\epsilon)\subseteq U$. There is a $K(\epsilon)$ such that for all $n\geq K(\epsilon)$, $x_n\in B(x,\epsilon)$. From these, we can choose a term that is not $x$. The definition of $a'$ is thus fulfilled, and so we are done
 \end{proof}
 \begin{theorem}
 Let $X,Y$ be metric spaces. Then, $f:X\rightarrow Y$ is continuous if and only if for every convergent sequence $x_n\in X$, the sequence $f(x_n)$ is convergent in $Y$ 
 \end{theorem}
 \begin{proof}
 For $x_n\rightarrow x$, consider $B(f(x),\epsilon)$. Then $f^{-1}(B(f(x),x))$ is an open neighbourhood of $x$. Thus, there is an integer $K(\epsilon)\in \N$ such that for all $n\geq K(\epsilon)$, $x_n\in f^{-1}(B(f(x),\epsilon))$. Thus, for all $n\geq K(\epsilon)$, $f(x_n)\in B(f(x),\epsilon))$, so $f(x_n)$ is convergent in $Y$.\\
 \\
 Conversely, assume that $x_n$ converges to $x$. Consider $x_1,x,x_2,x,x_3,x\cdots$. This converges to $x$. Then $f(x_1),f(x),f(x_2),f(x),\cdots$ converges by hypothesis. Since it has a constant subsequence equal to $f(x)$, the sequence itself converges to $f(x)$. The conclusion then follows from Proposition 11.
 \end{proof}
 \begin{definition}
 $X$ is Hausdorff if for every $x,y\in X$, $x\neq y$ there are open sets $U,V$ such that $x\in U$, $y\in V$, $U\cap V=\emptyset$.
 \end{definition}
 \begin{theorem}
 If $X$ and $Y$ are homeomorphic, and $X$ is Hausdorff, then $Y$ is Hausdorff.
 \end{theorem}
 \begin{proof}
 Consider $h:Y\rightarrow X$, a homeomorphism. Let $x,y\in Y$. Then $f(x)\neq f(y)$. There are disjoint open sets $U$ and $V$ in $X$ containing these 2 points separately. THe open sets $f^{-1}(U)$ and $f^{-1}(V)$ contain $x$ an $y$ respectively and are disjoint.
 \end{proof}
 \begin{remark}
 Hausdorff is a topological property.
 \end{remark}
 \begin{example}
 Consider $\C^n$ with the standard topology and $\C^n$ with the Zariski Topology. These are NOT homeomorphic. One is Hausdorff and the other is not.
 \end{example}
 \begin{definition}
 A topological space $X$ is not connected if there exist two open sets $U,V$ such that $U\cup V = X$ and $U\cap V=\emptyset$.
 \end{definition}
 \begin{example}
 $(-\infty,0)\cup (0,\infty)$ is not connected.
 \end{example}
 \noindent If a space is not ``not connected", then it is connected.
 \begin{example}
 $\Q$ is not connected.
 \end{example}
 \begin{proposition}
 \begin{enumerate}[(a)]
 \item
 If $A,B\subseteq X$ are disjoint subsets such that $A\cup B=X$ and neither of these sets contains a limit point of the other. Then, they form a separation of $X$. 
\item If $U,V$ form a separation of $X$ and if $Y\subseteq X$, $Y$ connected, then $Y$ lies entirely inside $U$ or $V$.
 \end{enumerate}
 \end{proposition}
 \begin{proof}
 \textbf{(a)}:\\
 \\
 Since neither contain limit points of the other, $\overline{A}=A$ and $\overline{B}=B$, so $A$ and $B$ are closed. Then, $A=X\setminus B$ and $B=X\setminus A$ are open, and form a separation.\\
 \\
 \textbf{(b):}\\
 \\
 Consider $\cap U$ and $Y\cap V$. $\overline{Y\cap U}\subseteq U$., and $\overline{Y\cap V}=V$. Thus, $\overline{Y\cap U}\cap \overline{Y\cap V}=\emptyset$, so $Y$ is entirely in $A$. 
 \end{proof}
 \begin{proposition}
 \begin{enumerate}[1)]
     \item The union of the collection of connected sets that share one point is connected. 
 \end{enumerate}
 \end{proposition}
 \begin{proof}
 Let us assume that $X_\alpha$, $\alpha\in A$ are connected, $x\in X_\alpha$ for all $\alpha$. Let us assume there is separation $U\cup V= \displaystyle\cup_{\alpha\in A}X_\alpha$. Then, there is a $y\in\displaystyle\bigcup_{\alpha\in A}X_\alpha$ such that $x,y$ are in different sets $U,V$. However, $y\in X_\alpha$ for some $\alpha$, and $U\cap X_\alpha$< $V\cap X_\alpha$ are open, disjoint and $(U\cap X_\alpha) \cup (V\cap X_\alpha)=X_\alpha$. $x$ is in one, $y$ is in the other. This then forms a separation of $X_\alpha$, and we get a contradiction.
 \end{proof}
\begin{theorem}
	If $X$ is a connected top. sp. and if $f:X\rightarrow{Y}$ is continuous, then $f(X)$ is connected.
	\end{theorem}
\begin{proof}
	By contradiction.  Assume that $f(X)$ has a separation.  Then there exist disjoint open sets $U,V\in f(X)$ such that $U\cup V=f(X)$.  Then
	\begin{equation*}
	f^{-1}(U\cup V) = f^{-1}(U)\cup{}f^{-1}(V) = (f^{-1}\circ{f})(X) = X
	\end{equation*}
	and $f$ being continuous implies that the preimages of $U$ and $V$ are both open, and furthermore $U\cap{V}=\varnothing\implies{}f^{-1}(U)\cap{}f^{-1}(V)=\varnothing$, so these sets form a separation of $X$, contradiction.
\end{proof}
\begin{proposition}
	\begin{enumerate}
		\item The union of connected sets that have one point in common is connected
		\item If $A$ and $B$ are spaces such that, $\bar{A}=B$, then $B$ is connected iff $A$ is connected.
		\item The product of connected spaces is connected in the product topology
	\end{enumerate}
\end{proposition}
\begin{proof}
	\textbf{1.} (Already done, see above) \\
	\textbf{2.} (Already done, see above) \\
	\textbf{3.}\\
	We start with the finite case:\\Let $X$, $Y$ be connected. Choose $x_0\in X$ and $y_0\in Y$.  Then $(X\times \{y_0\})\cup(\{x_0\}\times{Y})$ is connected, being the union of the connected spaces $X\times\{y_0\}$ and $\{x_0\}\times{Y}$, sharing $(x_0,y_0)$.\\Next, the union
	\begin{equation*}
	(X\times\{y_0\})\cup(\{x\}\times{Y})
	\end{equation*}
	is connected for any $x\in X$ by the same argument, since the two share $(x,y_0)$.  Hence
	\begin{equation*}
	X\times Y = \cup_{x\in X}((X\times\{y_0\})\cup(\{x_0\}\times{Y}))
	\end{equation*}
	is connected since all sets in the union share $(x_0,y_0)$.\\Now for the infinite case.   Let us consider:\\
	\begin{equation*}
	\prod_{\alpha\in I}(X_\alpha)
	\end{equation*} where $I$ is some infinite family, and where $X_\alpha$ is connected for all $\alpha\in I$.  Choose any point $(a_\alpha)_{\alpha\in I}$.  Consider sets of the form:
	\begin{equation*}
	X_{\alpha_1}\times{}X_{\alpha_2}\times\cdots\times X_{\alpha_n}\times \prod_{\beta\neq\alpha}(a_\beta)
	\end{equation*}
	They are connected, and contain $(a_\alpha)$.  Then
	\begin{equation*}
	\cup_{\alpha_1,\alpha_2,\cdots,\alpha_n\in I}(X_{\alpha_1}\times X_{\alpha_2}\times\cdots\times X_{\alpha_n}\times\prod_{\beta\neq\alpha}\{a_\beta\})
	\end{equation*}
	is connected.  We will show this set is dense in $\prod_{\alpha\in I}(X_\alpha)$.\\
	Let $(x_\alpha)_{\alpha\in I} \subset \prod X_\alpha$.  Let $U$ be an open neighborhood of $(x_\alpha)$ in the product topology.  The claim is that $U\cap A\neq\varnothing$.  Consider a basis element 
	\begin{equation*}
	U_{\alpha_1}\times U_{\alpha_2}\times\cdots\times U_{\alpha_n}\times\prod_{\beta\neq\alpha_i}(X_\beta)
	\end{equation*}
	be a basis element inside $U$ containing $(x_\alpha)$.  It contains the point
	\begin{equation*}
	\{x_{\alpha_1}\}\times\{x_{\alpha_2}\}\times\cdots\times\{x_{\alpha_N}\}\times\prod_{\beta\neq\alpha_i}\{a_\beta\}
	\end{equation*}  This point lies in
	\begin{equation*}
	X_{\alpha_1}\times X_{\alpha_2}\times\cdots\times X_{\alpha_n}\times\prod_{\beta\neq\alpha_i}\{a_\beta\}
	\end{equation*}
	Therefore, $A$ is connected, and the closure $\bar{A} = \prod_{\alpha\in I}X_\alpha$ so the latter is connected
\end{proof}
\begin{definition}
	A \textit{connected component} of a topological space $X$ is a maximal (under set inclusion) connected subspace
\end{definition}
\begin{theorem}
	Every topological space can be partitioned into connected components
\end{theorem}
\begin{proof}
	Define a relation on points in $X$: for all $x,y \in X$, let $x\sim{y}$ if $y$ is in the connected component of $x$.  Then this relation is in fact an equivalence relation, which partitions $X$ into equivalence classes, which are in fact connected components.  
\end{proof}
\begin{remark}
	The number of connected components of a topological space $X$ is a numerical invariant modulo homeomorphisms.
\end{remark}
\begin{definition}
	A topological space $X$ is locally connected if every point $x\in X$ and every open neighborhood $U(x)$ containing $x$, there is a connected open set $V$ such that $x\in V\subset U$. 
\end{definition}
\begin{proposition}
	A topological space $X$ is locally connected iff the connected components of every open set are open.
\end{proposition}
\begin{proof}
	Suppose the space $X$ is locally connected.  If $U$ is an open set, the for every $x\in U$, then there is a connected open set $V_x\subset U$ with $x\in V_x$.  Let $C$ be the connected component of $U$ containing $x$.  Then $V_x\subset C$ because if it weren't, then $C\subsetneq C\cup V_x\subset U$, contradicting the maximality of $C$. Thus, $C$ is exactly the union $C = \cup_{x\in C}V_x \subset U$, and is therefore open.  Conversely, let $X$ be such that the connected components of every open set are open.  Then, given an open set and a point $x\in U$, the connected component of $U$ containing $x$ is open, so $x\in V\in U$, which is exactly the requirement for local connectedness. 
\end{proof}
Not all connected spaces are locally connected
\begin{example}
	The comb
\end{example}
\begin{theorem}{connected sets in $\R$}
	A subset $U\subset\R$ is connected if and only if it is a point, interval, or $R$.
\end{theorem}
\begin{proof}
	If $A\subset R$ is none of these, then there are $a,b \in A$ and $c\not\in A$ such that $a<c<b$.  But then $(-\infty,c)\cup(c,\infty)$ is a separation of $A$.\\Conversely, assume $A$ is not a point or $\R$, so let it be an interval.  Set $A=U\cup V$, and $U\cap V=\varnothing$, where $U,V$ are both open.  Then $U,V$ are both closed.  Let $a\in U$, $b\in V$, $a<b$.  Define $c=\sup\{x\in U,x<b\}$ Then $c\in \bar{U}\cap\bar{V}$ (contradiction).
	\end{proof}
\begin{example}
	$S^1$ is connected: $f: \R\rightarrow S^1: f(t) = e^{it}$ is a continuous function mapping a connected top space, $\R$ onto $S^1$.
	However, $\R$ and $S^1$ are not homeomorphic: let $g$ be a homeomorphism between them.  Then consider $g(\R\setminus\{0\}) = S^1\setminus\{g(0)\}$, but $\R\setminus\{0\}$ is disconnected, while $S^1\setminus\{g(0)\}$ remains connected. (contradiction)
\end{example}
\subsection{September 29, 2022}
Recall:
\begin{definition}
	A topological space $X$ is connected if $$X=U\cup V\implies U=\varnothing\lor V=\varnothing$$ for any disjoint open sets $U,V$
\end{definition}
\begin{theorem}
	The connected subsets of $\R$ are exactly \begin{itemize}
		\item $\R$
		\item $\{x\}$
		\item Open and/or closed intervals in $\R$.
	\end{itemize}
\end{theorem}
\begin{theorem}
	Let $f: [a,b]\rightarrow [a,b]$ be a continuous functions.  Then there is a fixed point in $[a,b]$ under $f$.  i.e. there exists some $c\in[a,b]$ such that $f(c) = c$
\end{theorem}
\begin{proof}
	Consider the function $g: [a,b]\to\R$ given by $$g(x)=f(x)-x$$Then $g$ is continuous, so $g([a,b])$ is a connected set.  We have $g(a)\geq0$ and $g(b)\leq 0$.  Thus, there is some point $c\in[a,b]$ at which $g(c)=0$. i.e. $f(c)-c=0$ or $f(c)=c$
\end{proof}
\begin{theorem}[Borsuk-Ulam]
	Given a continuous function $f: S^1\to\R$, there exist two antipodal points $z,-z$ such that $f(z)=f(-z)$.
\end{theorem}
\begin{proof}
	$S^1$ is connected.  $f(S^1)$ is therefore connected, as $f$ is continuous.  Consider another function: $$g: S^1\to\R\;\;  z\xrightarrow{g}f(z)-f(-z)$$Then $$g(z)=-g(-z)$$, but then there must be some $c$ such that $g(c)=0$ 
\end{proof}
\begin{theorem}
	Given two regions in the plane, there is a line that cuts them simultaneously into two equal halves.
\end{theorem}
\begin{proof}
	Pick any point in $\R^{2}$ and parameterize in $S^1$ (directed) lines through that point : $L(\theta)$.  Then there are lines $V_1(\theta)$ and $V_2(\theta)$ , which are perpendicular to $L(\theta)$ and which divide the first and second regions in half, respectively.  Let $v_1(\theta)$ and $v_2(\theta)$ be the points at which $V_1$ and $V_2$ intersect $L(\theta)$.  Then $$v_1(\theta)-v_2(\theta)=-(v_1(-\theta)-v_2(-\theta))$$, so by the previous theorem there is some $\theta\in S^1$ which makes these lines the same.
\end{proof}
\begin{definition}
	A topological space $X$ is called "path connected" if for every $x,y\in X$, there is a continuous function $f:[0,1]\to X$ such that $f(0)=x$ and $f(1)=y$.
\end{definition}
\begin{theorem}
	Every path connected space is connected
\end{theorem}
\begin{proof}
	Fix $x_0\in X$.  Then $$X=\cup_{\substack{f(0)=x_0\\f(1)=y	}}(f_y([0,1]))$$  But the union of connected spaces sharing a single point is connected.
\end{proof}
In general, if $X$ is a topological space, we say that $$x\sim y$$ if there is a path from $x$ to $y$.
\begin{proposition}
	Let $X$ be a topological space.  Let $\sim$ be a relation defined on points of $X$ such that $x\sim y$ if there exists a path from $x$ to $y$.  Then $\sim$ is an equivalence relation.
\end{proposition}
\begin{proof}
	\begin{itemize}
		\item $x\sim y$ implies the existence of a path $f: [0,1]\rightarrow X$ such that $f$ is continuous, $f(0)=x$ and $f(1)=y$, but then $g: [0,1]\rightarrow X$ defined by $g(t) = f(1-t)$ is also continuous, and $g(0)=f(1)=y$, and $g(1)=f(0)=x$, so $g$ is a path from $y$ to $x$.  Hence, $y\sim x$
		\item $x\sim x$, because $f(t) = x$ is a path.
		\item Let $x\sim y$ and $y\sim z$, so there exist continuous functions, $f,g: [0,1]\rightarrow X$ such that $f(0)=x$, $f(1)=g(0)=y$, $g(1)=z$, then let $$h(t) := \begin{cases}
		f(zt), t\in[0,\frac{1}{2}]\\
		f(zt-1), t\in[\frac{1}{2},1]
		\end{cases}$$.	But then $h$ is a path from $x$ to $z$, whence $x\sim z$\end{itemize}
\end{proof}
\begin{remark}
	Topological spaces are partitioned into path-connected components by $\sim$.
\end{remark}
\begin{theorem}
	\begin{enumerate}[i.]
		\item The union of path-connected topological spaces sharing a single point is path-connected.
		\item The product of path-connected topological spaces is path-connected.
		\end{enumerate}
\end{theorem}
\begin{proof}
	Let $X_\alpha\,,\alpha\in A$ be path connected spaces with $p\in\cap_{\alpha\in A}X_\alpha$.  Let $r,s$ be points in $\cup_{\alpha\in A}X_\alpha$.  Then there is some $\beta_1,\beta_2$ in $A$ such that $r\in X_{\beta_1}$ and $s\in X_{\beta_2}$.  Then there exist paths from $r$ to $p$ and $p$ to $s$, so simply concatenate them.\\
	(b) Pick two points $x,y$ in $X:=\prod_{\alpha\in A}X_\alpha$.  Then for every $\alpha\in A$, there exists a continuous function $f_\alpha: [0,1]\to X_\alpha$ such that $f$ is continuous, $f_\alpha(0)=\pi_\alpha(x)$ and $f_\alpha(1)=\pi_\alpha(y)$, but then by the universal property there exists a unique function $f:[0,1]\to X$ such that $f(0)=x$ and $f(1)=y$, but according to the product topology then $f$ must be continuous, i.e. a path.
\end{proof}
\begin{definition}
	A topological space is called "locally path-connected" if every open neighborhood of any point $x$ contains a path-connected open set, itself containing $x$.
\end{definition}
\begin{proposition}
	A space is locally path-connected if and only if the path components of every open set are open.
\end{proposition}
\begin{proof}
	Assume all of the path components of $X$ are open.  For any open neighborhood $U$ of $x$, choose the path component of $U$ containing $x$ as $V$.  Conversely, let $X$ be locally path-connected.  Let $U$ be open, and let $C$ be a path component of $U$.  For every $x\in C$, choose $V_x$ to be open, path connected, and $x\in V_x\subset U$.  Then $V_x\subset C$, and $C\subset\cup_{x\in C}V_x$.
\end{proof}
\begin{example}[Deleted Comb]
	Not all connected spaces are path-connected
\end{example}
\begin{proof}[Proof (of example)]
Let the "deleted comb" $C$ be defined: $$C:=((0,1] \times \{0\}) \cup ((\{0\}\times (0,1])) \cup (\bigcup_{n=1}^{\infty}(\{\frac{1}{n}\}\times [0,1]))$$
Then $C$ is connected, but not path-connected.
\end{proof}
\begin{example}[Comb]
	The comb is locally path-connected but not path-connected.
\end{example}
\begin{definition}[Compact Spaces]
	A topological space $X$ is called "compact" if it is Hausdorff and if every open cover of $X$ admits a finite subcover.
\end{definition}
\subsection{October 11, 2022}
\begin{theorem}[Lebesgue's Number Theorem]
	Let $X$ be a compact metric space, and let $U$ be an open cover of $X$.  Then there exists a real $\delta>0$ with the property that, for any open set $A\subset X$ with $\textsf{diam}(A)\leq\delta$, then there is some $U_i\in U$ containing $A$.
\end{theorem}
\begin{corollary}
	Let $f: X\rightarrow Y$ be a continuous function of metric spaces, and let $X$ be compact.  Then $f$ is uniformly continuous.
\end{corollary}
\begin{proof}
	Let $\epsilon>0$.  For $x\in X$, there is $\delta_x>0$ such that if $d_X(x,y)<\delta_x$ then $d_Y(f(x),f(y))<\frac{\epsilon}{2}$.  Let $\delta>0$ be the Lebesgue number of the open cover of $X$ by the balls $B(x,\delta_x)$.  If $x,y$ are such that $d(x,y)<\delta$, then there is $B(x_o,\delta_{x_o})$ such that $x,y\in B(x_o,\delta_{x_o})$ then $$d(f(x),f(y))<d(f(x),f(x_o))+d(f(x_o),f(y))<\frac{\epsilon}{2}+\frac{\epsilon}{2}=\epsilon$$
\end{proof}
\begin{theorem}[AM-GM Inequality]
	If $x_1,x_2,..,x_n >0$, then 
	\begin{equation}
	\frac{1}{n}\sum_{i=1}^n(x_i) \geq \sqrt[n]{\prod_{i=1}^n(x_i)}
	\end{equation}, with equality if and only if $x_i=x_j$ for all $1\leq i\leq j\leq n$
\end{theorem}
\begin{proof}
	The inequality is invariant under simultaneous multiplication of $x_i$ by $\lambda >0$, so we assume without loss of generality that $\sum_{i=0}^n(x_i)=1$.  Furthermore, the result is trivial if $x_i=0$ for any $1\leq i\leq n$, so we consider $$X:=\{(x_1,x_2,...,x_n)|\sum_{i=0}^n(x_i)=1\land x_i\geq0\:\forall 1\leq i\leq n\}$$.  $X$ is closed and bounded, and thus it is compact.  For any pair $x_i, x_j\in X$, $x_i<x_j$, let $\epsilon<x_j-x_i$.  Then $$(x_i+\epsilon)(x_j-\epsilon) = x_ix_j + \epsilon(x_j-x_i-\epsilon) > x_ix_j$$.\\ Now, let $f:X\rightarrow \R$ be defined by $$f(x_1,...,x_n) := \sqrt[n]{x_1x_2\cdots x_n}$$.  If not all $x_i$ are equal, choose $x_i<x_j$ and replace them with $x_i+\epsilon$, $x_j+\epsilon$, respectively.  The sum $$x_1+...+x_i+\epsilon+...+x_j-\epsilon+...+x_n=1$$But: $$f(x_1,...,x_i+\epsilon,...,x_j-\epsilon,...,x_n)>f(x_1,...,x_n)$$So $f$ does not attain a maximum on $X$ unless all points are equal.  So the max is $(\frac{1}{n},\frac{1}{n},...,\frac{1}{n})$, completing the proof.
\end{proof}
\begin{example}[Balkam Math Olympiad 1984]
	Let $\alpha_1,\alpha_2...,\alpha_n>0$ and $\sum_{i=0}^n(\alpha_i)=1$, prove that $$\sum_{i=0}^n (\frac{\alpha_i}{2-\alpha_i})\geq\frac{n}{2n-1}$$
\end{example}
\begin{theorem}
	If $X,Y$ are compact spaces, then $X\times Y$ is compact.
\end{theorem}
\begin{proof}
	Consider an open cover $U$ of $X\times Y$.  Fix $x_o\in X$, and note that $U$ is also a cover for $\{x_o\}\times Y$.  This has a subcover $$U':=\{U_1^{x_o},U_2^{x_o},...,U_n^{x_o}\}$$.  By the tube lemma, there is $W_{x_o}\subset X$ open such that $$\{x_o\}\times Y\subset W_{x_o}\times Y\subset U'$$.  Vary $x_o$ to obtain the open cover $(W_{x_o})_{x_o\in X}$ of $X$.  There are finitely many $x_o$'s, call them $x_1,...,x_n$, such that $$\bigcup_{i=1}^n(W_{x_i})=X$$.  Each $W_{x_i}\times Y$ is covered by finitely many $u\in U$.  So the union $\bigcup_{i=0}^n(W_{x_i}\times Y)$ is itself covered by finitely many elements in $U$, so $X\times Y$ has a finite subcover, finishing the proof.
\end{proof}
 \end{document}
