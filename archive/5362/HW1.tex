\documentclass{article}
\usepackage{amsmath}
\usepackage{amssymb}
\usepackage{amsthm}
\usepackage{mathrsfs}

\newtheorem{theorem}{Theorem}[section]
\newtheorem{lemma}[theorem]{Lemma}
\newtheorem{proposition}[theorem]{Proposition}
\newtheorem{corollary}[theorem]{Corollary}
\newtheorem{definition}[theorem]{Definition}
\newtheorem{construction}[theorem]{Construction}
\newtheorem{example}[theorem]{Example}
\newtheorem{notation}[theorem]{Notation}
\newtheorem{remark}[theorem]{Remark}
\newtheorem{chunk}[theorem]{}

\DeclareMathOperator{\Z}{\mathbb{Z}}
\DeclareMathOperator{\Q}{\mathbb{Q}}
\DeclareMathOperator{\N}{\mathbb{N}}
\DeclareMathOperator{\C}{\mathbb{C}}
\DeclareMathOperator{\scrO}{\mathscr{O}}

\begin{document}
\title{MATH 5362 Homework}
\author{Orin Gotchey}
\maketitle
\section{Homework 1}
\begin{lemma}
If $\alpha^n\in\scrO_{K}$ for some natural number $n\in\N$ and field $K$, then $\alpha\in\scrO_{K}$
\end{lemma}
\begin{proof}
Let $\alpha,\;n\;,K$ be as stated.  Then there exists some $g(x)\in\Z[x]$, written
\begin{equation}
g(x) = \Sigma_{i=0}^{deg(g)}(\beta_ix^i)
\end{equation}
where $\beta_i\in\Z$, and such that $g(\alpha^n)=0$.  But $\alpha$ must be a root of:
\begin{equation}
h(x) := g(x^n) \in\Z[x]
\end{equation}
\end{proof}
\begin{proposition}{Problem 1}

$\theta := \frac{10^{\frac{2}{3}}-1}{\sqrt{-3}}$ is an algebraic integer.
\end{proposition}
\begin{proof}
\begin{equation}
\omega := (-3\cdot\theta^2)+1 = 10^{\frac{4}{3}} - 2*10^{\frac{2}{3}}
\end{equation}
Since $\scrO$ is a ring, and since $10^{\frac{2}{3}}$ is a root of $f(x) = x^3-100$, and is therefore an algebraic integer, we get: $\omega\in\scrO$.  That is, there exists some $f\in\Z[x]$ such that $f(\omega)=0$.  But then $f(-3\cdot\theta^2)=0$ gives rise to another $g(x)\in\Z[x]$ such that $g(\theta)=0$.  So $\theta\in\scrO$, as required;
\end{proof}
\begin{lemma}
	Let $m\in\N$.  Then $\sqrt{m}$ is irrational or an integer.
\end{lemma}
\begin{proof}
	Assume $\sqrt{m}$ is a non-integer rational.  Then there exist some $a
	\neq{}b\in\Z$ such that $\gcd(a,b)=1$ and $\sqrt{m}=\frac{a}{b}$.  Thus, $m = \frac{a^2}{b^2}$, and we know from the Fundamental Theorem of Arithmetic that $\gcd(a^2,b^2) = 1$.  This forces $a^2=b^2$ or $a=b$, contradiction.
\end{proof}
\begin{proposition}{Problem 2}\\
For a given $m\in\N$, the quantity $\alpha:= \frac{\sqrt{m}+1}{\sqrt{2}}$ is an algebraic integer iff $m$ is odd.
\end{proposition}
\begin{proof}
First note that $m$ is odd iff the quantity $(m+1)^2\equiv_{4}0$, which in turn is true iff the polynomial
\begin{equation*}
f(x) = x^4 - (m+1)x^2 - (\frac{3}{4}(m+1)^2 + 3m) = 0
\end{equation*}
is an element of $\Z[x]$.   $\alpha$ is a root of $f(x)$ (check), completing the forward direction.  Now assume that $\alpha$ is an algebraic integer.  Then let $p(x)\in\Z[x]$ be the monic minimal polynomial for $\alpha$.  It holds that $p|f$ in $\Q[x]$.  We will show that this forces $f(x)\in\Z[x]$, completing the proof.\\We know that $1\leq\deg(p)\leq4$, so three cases for $\deg(p)$:
\begin{enumerate}
	\item $\deg(p)=1$ would imply that $\alpha\in\Z$, but the constant term of $f$ must be $\alpha*p_0$ for some $p_0\in\Z$, whence it follows $f(x)\in\Z[x]$.
	\item $\deg(p)=2$.  Let $p(x)=x^2+p_1x+p_0$.  Then there exist $a,b,c\in\Q$ such that
	\begin{equation*}
	(x^2+p_1x+p_0)(ax^2+bx+c) = f(x)
	\end{equation*}
	This implies that $a=0$, and then $b=-p_1$ combined with the fact that $p_0+c+pb_1\in\Z$ grants us that $a,b,c\in\Z$, and thus $f(x)\in\Z[x]$
	\item $\deg(p)=3$.  Let $p(x) = x^3+p_2x^2+p_1x+p_0$ with the $p_i\in\Z$, and then there is some $a\in\Q$ such that $p(x)(x+a)=f(x)$.  But since $f$ has no term of degree $3$, $p_2+a=0$, but then $x-a\in\Z[x]$, so $f(x)\in\Z[x]$. 
	\item $\deg(p)=4$.  Then $p(x)=f(x)$, whence it follows $f(x)\in\Z[x]$.
\end{enumerate}
\end{proof}
\begin{proposition}
	Let $\alpha := (\frac{1+\sqrt{2}}{9})^\frac{1}{3} + (\frac{1-\sqrt{2}}{9})^\frac{1}{3}$.  Then $\alpha/729$ is an algebraic integer
\end{proposition}
\begin{proof}
	$\alpha$ satisfies the equation $\alpha^3+3^\frac{1}{3}\alpha-\frac{2}{9} = 0$.  That is,
	\begin{align*}
	&3^{\frac{1}{3}}\alpha = -\alpha^3 + \frac{2}{9}\\
	&3\alpha^3 = -\alpha^9+\frac{2}{3}\alpha^6-\frac{4}{27}\alpha^3+\frac{8}{729}\\
	&729\alpha^9-486\alpha^6+2305\alpha^3-8=0\\
	&729^{10}(\frac{\alpha}{729})^9 + ... = 0\\
	\end{align*}
	Thus, $\frac{\alpha}{729}$ is an algebraic integer.
\end{proof}
\begin{proposition}
	The minimal polynomials of $\alpha := \frac{1+i}{\sqrt{2}}$ over $\Q,\Q(i),\Q(\sqrt{2})$, respectively, are:
	\begin{itemize}
		\item $f(x)=x^4+1$
		\item $g(x)=x^2-i$	
		\item $h(x)=x^2-\sqrt{2}x+1$
	\end{itemize}
\end{proposition}
\begin{proof}
	Clearly, these are monic polynomials over their respective fields of which $\alpha$ is a root.  Since $\alpha$ is a primitive 8th root of unity, and $f$ is the 8-th cyclotomic polynomial, it is irreducible.  $g$ and $h$ are irreducible because $\alpha\not\in\Q(i)$ and $\alpha\not\in\Q(\sqrt{2})$.
\end{proof}
\begin{proposition}
	$\Q(\sqrt{2},\sqrt{3},\sqrt{6}) = \Q(\sqrt{2}+\sqrt{3}+\sqrt{6})$ and also $[\Q(\sqrt{2},\sqrt{3},\sqrt{6}) : \Q]=8$.
\end{proposition}
The minimal polynomials of $\sqrt{2}$ and $\sqrt{3}$ are $x^2-2$ and $x^2-3$, respectively.  Since $\frac{-2\sqrt{2}}{-2\sqrt{3}}\not\in\Q$, then any $c\in\Q$ gives us a primitive element, $\theta=c\sqrt{2}+\sqrt{3}$ generating $\Q(\sqrt{2},\sqrt{3})$.  Let $c=1$. Then $\Q(\sqrt{2}+\sqrt{3}) = \Q(\sqrt{2},\sqrt{3})$.  Also, $[\Q(\sqrt{2},\sqrt{3}) : \Q] = 4$, because $\sqrt{3}$ is not a $\Q$-linear combination of $\{1,\sqrt{2}\}$.  The conjugates of $\sqrt{2}+\sqrt{3}$ are just $\pm\sqrt{2}\pm\sqrt{3}$, and the other conjugate of $\sqrt{5}$ is $-\sqrt{5}$. We need to find a $c\in\Q$ such that
\begin{equation*}
c\neq \frac{(\sqrt{2}+\sqrt{3})\pm\sqrt{2}\pm\sqrt{3}}{2\sqrt{5}}
\end{equation*}
again, $c=1$ is suitable, so it follows that\
\begin{equation*}
\Q(\sqrt{2}+\sqrt{3}+\sqrt{5}) = \Q(\sqrt{2},\sqrt{3},\sqrt{5})
\end{equation*}
Now, all we need to show is that $\sqrt{5}\not\in\Q(\sqrt{2},\sqrt{3})$.  This would require some $a,b,c\in\Q$ such that
\begin{multline*}
\sqrt{5} = a\sqrt{2}+b\sqrt{3}+c\sqrt{6}\\
5 = 2a^2+3b^3+6c^2+ab\sqrt{6}+ac\sqrt{12}+bc\sqrt{18}\\
ab\sqrt{6}+2ac\sqrt{3}+3bc\sqrt{2}\in\Q
\end{multline*}
But $\sqrt{2}$,$\sqrt{3}$,and $\sqrt{6}$ are linearly independent over $\Q$.  If they weren't, then there would be $\alpha,\beta\in\Q$:
\begin{equation*}
\sqrt{6} = \alpha\sqrt{2} + \beta\sqrt{3}
\end{equation*}
\begin{align*}
&6 = 2\alpha^2+3\beta^2+2\alpha\beta\sqrt{6}\\
&\sqrt{6}\in\Q
\end{align*}
(contradiction, see above).  Thus: $[\Q(\sqrt{2},\sqrt{3},\sqrt{5}:\Q)] = 8$
\end{document}