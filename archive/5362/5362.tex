\documentclass{article}
\usepackage[utf8]{inputenc}
\usepackage{amsmath}
\usepackage{amssymb}
\usepackage{amsfonts}
\usepackage{float}
\usepackage{amsthm}
\usepackage{graphicx}
\usepackage{fullpage}
\usepackage{color}
\usepackage{enumerate}[shortlabels]
\usepackage{hyperref}
\hypersetup{
colorlinks=true,
linkcolor=blue,
urlcolor=red,
linktoc=all
}
\definecolor{dg}{rgb}{0.0, 0.5, 0.0}
\newcommand{\lgs}[2]{\left(\frac{#1}{#2}\right)}
\newcommand{\R}{\mathbb{R}}
\newcommand{\C}{\mathbb{C}}
\newcommand{\la}{\left\langle}
\newcommand{\ra}{\right\rangle}
\newcommand{\air}{\mathcal{O}_K}
\newcommand{\Q}{\mathbb{Q}}
\newcommand{\Z}{\mathbb{Z}}
\newcommand{\D}{\mathbb{D}}
\newcommand{\HH}{\mathbb{H}}
\newcommand{\N}{\mathcal{N}}
\newtheorem{theorem}{Theorem}[subsection]
\newtheorem{remark}{Remark}[subsection]
\newtheorem{cor}{Corollary}[subsection]
\newtheorem{conj}{Conjecture}[subsection]
\newtheorem{example}{Example}[subsection]
\newtheorem{proposition}{Proposition}[subsection]
\newtheorem{lemma}{Lemma}[subsection]
\newtheorem{definition}{Definition}[subsection]
\title{MATH 5326-Algebraic Number Theory Final Exam}
\author{Orin Gotchey}

\begin{document}

\maketitle

\section*{Problem I}
Let $m\equiv_4 1$ be a squarefree integer. Then the set of all elements $K=\Q(\sqrt{m})$ which are integral over $\Z[\sqrt{m}]$ is equal to $\Z[\frac{1+\sqrt{m}}{2}]$
\begin{proof}
First, let $\alpha=\frac{a+b\sqrt{m}}{2}$ for some integers $a,b\in\Z\:a\equiv_2b$.  We demonstrate that $\alpha$ is algebraic over $\Z$: \begin{align*}
\alpha^2 &=\frac{a^2+2ab\sqrt{m}+b^2m}{4}\\
\alpha^2-a\alpha&=\frac{a^2+2ab\sqrt{m}+b^2m-2ab\sqrt{m}-2a^2}{4}\\
&=\frac{-a^2+b^2m}{4}
\end{align*}
Now, $$a\equiv_2b\implies a^2\equiv_4b^2\equiv_4 b^2m\implies \frac{-a^2+b^2m}{4}\in\Z$$ Thus, $\alpha$ is a root of the monic integer polynomial $$f(x)=x^2-ax+\frac{a^2-b^2m}{4}\in\Z[x]$$\\Conversely, we assume that $\alpha=\frac{a}{b}+\frac{c}{d}\sqrt{m}$ is algebraic over $\Z$, with $b\neq0\neq d,\;\gcd(a,b)=1,\;\gcd(c,d)=1$.  Thus, $\alpha$ is the root of some monic integer polynomial $$f(x)=x^2+\gamma_1x+\gamma_0$$
for some $\gamma_1,\:\gamma_0\in\Z$.  Substituting $\alpha$ for $x$,
\begin{equation}\label{eq1}f(\alpha)=\frac{a^2}{b^2}+\frac{2ac}{bd}\sqrt{m}+m\frac{c^2}{d^2}+\gamma_1(\frac{a}{b}+\frac{c}{d}\sqrt{m})+\gamma_0=0\end{equation}
Now, $1$ and $\sqrt{m}$ are linearly independent over $\Z$, so we separate:
\begin{align}
\label{eq:2}
\frac{2ac}{bd}\sqrt{m}+\gamma_1\frac{c}{d}\sqrt{m}&=0\\
\label{eq:3}
\frac{2ac}{bd}+\frac{\gamma_1bc}{bd}&=0\\
\label{eq:4}
2ac+\gamma_1bc=c(2a+\gamma_1b)&=0\\
\label{eq:5}
\text{(and...)}\ \frac{a^2}{b^2}+m\frac{c^2}{d^2}+\gamma_1\frac{a}{b}+\gamma_0&=0
\end{align}
In view of (\ref{eq:4}), either $c=0$ or $\gamma_1=\frac{-2a}{b}$.  In the former case, we get $\alpha\in\Q$ the root of some monic integer polynomial, so $\alpha\in\Z\subseteq\Z[\frac{1+\sqrt{m}}{2}]$ (in which case we'd be done).  On the other hand, if $a=0$, then $\frac{mc^2}{d^2}\in\Z$.  However, $m$ is squarefree, so $d\,|\,c$ and $\alpha\in\Z[\sqrt{m}]$, and we're done.  Thus, we may assume that $c\neq0\neq a$ and $\gamma_1=\frac{-2a}{b}$.  This implies that $b|2a$.  
\begin{enumerate}[C{a}se 1]
	\item In the case that $b$ is an odd integer, then $b|a$ since $2$ is prime.  $b$ and $a$ were chosen such that $\gcd(b,a)=1$, thus $b=\pm1$.  Rewriting (\ref{eq:5}),
	$$\frac{-a^2}{b^2}+m\frac{c^2}{d^2}\in\Z$$
	$$-a^2+m\frac{c^2}{d^2}\in\Z$$
	$$\therefore (d^2)|(mc^2)$$ $m$ is still squarefree, so $d|c$, and thus $\alpha\in\Z$, completing this case.
	\item If $b$ is even: $$(\exists x\in\Z\::2x=b)\therefore 2x|2a\therefore x|a\therefore x|\gcd(a,b)$$Thus, $x$ is a unit, so we can assume WLOG that $b=2$.  
	\end{enumerate}
So, $a$ is an odd integer and $b=2$.
\begin{align*}
\frac{a}{4}^2+m\frac{c^2}{d^2}+\frac{-2a^2}{4}+\gamma_0&=0\\
\frac{-a^2}{4}+m\frac{c^2}{d^2}&\in\Z\\
\frac{4mc^2}{d^2}&\in\Z\\
\frac{m(2c)^2}{d^2}&\in\Z\\
d^2 &|\,m(2c)^2\\
\text{m squarefree}\therefore d^2 &|\,(2c)^2\\
d\,&|\,2c
\end{align*}
Arguing in a way symmetric to that above: if $d$ were odd, then $d|c$.  Since $\gcd(d,c)=1,\ d=\pm1$  In that case, $$\frac{-a^2}{4}+mc^2\in\Z$$
$$\frac{-a^2}{4}\in\Z$$, which implies that $\alpha\in\Z\subseteq\Z[\frac{1+\sqrt{m}}{2}]$, and so we may assume that $d$ is even.  An argument perfectly symmetric to that above shows that therefore $d=2$, and with $\gcd(c,d)=1$, we see that $c$ is odd.
$$\alpha=\frac{a+c\sqrt{m}}{2}=\frac{a-c}{2}+c\frac{1+\sqrt{m}}{2}$$Recall that $a\equiv_2 c$, so $\alpha\in\Z[\frac{1+\sqrt{m}}{2}]$, completing the proof.
\end{proof}
\section*{Problem II}
Let $K=\Q(\theta)$ where $\theta$ is a root of $f(x):=x^6+2x^2+2=0$. Let $\alpha:=\theta^4+\theta^2=\theta^2(1+\theta^2)$.  The minimal polynomial of $\alpha$ is $g(x)= $
\begin{proof}
	\begin{align*}
	\alpha&=\theta^4+\theta^2\\
	\alpha^2&=\theta^8+2\theta^6+\theta^4\\
	&= (\theta^2+2)\theta^6+\theta^4\\
	&= (\theta^2+2)(-2\theta^2-2)+\theta^4\\
	&=-2\theta^4-6\theta^2-4+\theta^4\\
	&=-\theta^4-6\theta^2-4\\
	\alpha^3&=(-\theta^4-6\theta^2-4)(\theta^4+\theta^2)\\
	&=-\theta^8-\theta^6-6\theta^6-6\theta^4-4\theta^4-4\theta^2\\
	&=-\theta^8-7\theta^6-10\theta^4-4\theta^2\\
	&=-\theta^6(\theta^2+7)-10\theta^4-4\theta^2\\
	&=(2\theta^2+2)(\theta^2+7)-10\theta^4-4\theta^2\\
	&=(2\theta^4+16\theta^2+14)-10\theta^4-4\theta^2\\
	&=-8\theta^4+12\theta^2+14	
	\end{align*}Observe: $\alpha,\;\alpha^2,\;\alpha^3$ are $\Z$-linear combinations of $\{\theta^2,\theta^4,1\}$.
	
	Namely,
	\[
	\begin{bmatrix}
	1&1&0\\
	-1&-6&-4\\
	-8&12&14
	\end{bmatrix}
	\begin{bmatrix}
	\theta^4\\
	\theta^2\\
	1
	\end{bmatrix}
	=
	\begin{bmatrix}
	\alpha\\
	\alpha^2\\
	\alpha^3
	\end{bmatrix}
	\]We are going to perform a kind of row reduction on the coefficient matrix, keeping track of the effects on the $\alpha^i$'s.
	\[
	\left[
	\begin{array}{@{}ccc|c@{}}
	1 & 1 & 0 & \alpha\\
	0&-5&-4&\alpha^2+\alpha\\
	0&20&14&\alpha^3+8\alpha
	\end{array}
	\right]\]
	\[
	\left[
	\begin{array}{@{}ccc|c@{}}
	1&1&0&\alpha\\
	0&-5&-4&\alpha^2+\alpha\\
	0&0&-2&\alpha^3+8\alpha+4(\alpha^2+\alpha)
	\end{array}
	\right]
	\]Thus, we see that $$\alpha^3+4\alpha^2+12\alpha=-2$$, i.e.$$\alpha^3+4\alpha^2+12\alpha+2=0$$.  Thus, $\alpha$ is a root of $g(x):=x^3+4x^2+12x+2$.  This polynomial is monic, and is irreducible by Eisenstein.
\end{proof}
\section*{Problem III}
	Let $p\equiv_4 3$ be a prime, and let $K=\Q(\sqrt{p})$.  It is known that $h_K$ is odd.  As a result, there exist integers $a,b\in\Z$ such that $a^2-pb^2=(-1)^{\frac{p+1}{4}}2$
\begin{proof}
	Consider the ideal $<2,1+\sqrt{p}>$ in $K$.  Then $<2,1+\sqrt{p}>=<2,1+\sqrt{p}-2\sqrt{p}>=<2,1-\sqrt{p}>$
	\begin{align*}
	<2,1+\sqrt{p}>^2\,=&\\
	=&<2,1+\sqrt{p}><2,1-\sqrt{p}>\\
	=&<4,1-p>\\
	=&<2>
	\end{align*}The last equality follows because $1-p\equiv_4 2\implies \exists m\in\Z:\;(1-p)+4m=2$, and both $4$ and $1-p$ are generated by $2$.  Therefore, the ideal $<2,1+\sqrt{p}>^2$ is principal.  The order of the class $[<2,1+\sqrt{p}>]$ then divides both two and $h_K$(odd), so it must be $1$.  Thus, $<2,1+\sqrt{p}>$ is principal.  This means that there exist $a,b\in\Z$ such that $<2,1+\sqrt{p}>=<a+b\sqrt{p}>$.  Then,
	$$<2>=<2,1+\sqrt{p}>^2\,=<a+b\sqrt{p}>^2\,=<a+b\sqrt{p}><a-b\sqrt{p}>=<a^2-pb^2>$$
	Since $a^2-pb^2\in\Z$ and $2$ both generate the same ideal, they must differ by a unit in $\Z$.  Thus, $$a^2-pb^2=\pm{2}$$.  We know that $a^2,b^2\equiv_8 1\;\text{or}\;4$, and that $p\equiv_8 3+ 4(\frac{p+1}{4})$, so we break into cases:
	\[
	\begin{array}{cccc}
		[a^2]_8&[b^2]_8&[p]_8&[a^2-bp^2]_8\\
		1&1&3&6\\
		1&1&7&2\\
		1&4&3&5\\
		1&4&7&5\\
		4&1&3&1\\
		4&1&7&5\\
		4&4&3&0\\
		4&4&7&0
	\end{array}
	\]To recap: if $p\equiv_83$, then $a^2-bp^2\equiv_86$, so $a^2-bp^2=(-2)$.  Conversely, if $p\equiv_87$, then $a^2-bp^2\equiv_82$, so $a^2-bp^2=2$.  Combining these two cases into one equation, we see that $$a^2-pb^2=(-1)^\frac{p+1}{4}2$$
\end{proof}
\end{document}
