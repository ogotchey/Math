\documentclass{article}
\usepackage[utf8]{inputenc}
\usepackage{amsmath}
\usepackage{amssymb}
\usepackage{amsfonts}
\usepackage{float}
\usepackage{amsthm}
\usepackage{graphicx}
\usepackage{fullpage}
\usepackage{color}
\usepackage{enumerate}[shortlabels]
\usepackage{hyperref}
\hypersetup{
    colorlinks=true,
    linkcolor=blue,
    urlcolor=red,
    linktoc=all
}
\definecolor{dg}{rgb}{0.0, 0.5, 0.0}
\newcommand{\R}{\mathbb{R}}
\newcommand{\C}{\mathbb{C}}
\newcommand{\N}{\mathbb{N}}
\newcommand{\air}{\mathcal{O}_K}
\newcommand{\Q}{\mathbb{Q}}
\newcommand{\Z}{\mathbb{Z}}
\newcommand{\D}{\mathbb{D}}
\newcommand{\HH}{\mathbb{H}}
\newtheorem{theorem}{Theorem}
\newtheorem{remark}{Remark}
\newtheorem{cor}{Corollary}
\newtheorem{example}{Example}
\newtheorem{proposition}{Proposition}
\newtheorem{lemma}{Lemma}
\newtheorem{definition}{Definition}
\title{Algebraic Number Theory Notes - Spring 2022}
\author{JJ Hoo}
\date{January 2022}

\begin{document}

\maketitle
\tableofcontents
\newpage
\section{Algebraic Numbers and Algebraic Integers}
\subsection{August 26, 2022}
\begin{definition}
A number $\alpha\in\C$ is called \underline{algebraic number} if there exists $0\neq f(x)\in\Q[x]$ such that $f(\alpha)=0$. In other words, $\alpha$ is algebraic over $\Q$.
\end{definition}
\noindent Note that if $\alpha$ is an algebraic number, then there exists $0\neq g(x)\in\Z[x]$ such that $g(\alpha)=0$\footnote{This is done by multiplying $f(x)$ through by the LCM}.
\begin{definition}
$\alpha$ is an \underline{algebraic integer} if there exists $0\neq f(x)\in \Z[x]$ where $f$ is monic, such that $f(\alpha)=0$.
\end{definition}
\begin{proposition}
Every algebraic integer is an algebraic number. On the other hand, the converse ise false.
\end{proposition}
\begin{example}
Let $\alpha=\frac{\sqrt{2}}{3}$. This is an algebraic number, but NOT an algebraic integer.\\
\\
Consider $f(x)=9x^2-2\in \Z[x]\subseteq \Q[x]$. Then, $f(\alpha)=0$, so $\alpha$ is an algebraic number.\\
\\
Now, let $g(x)\in \Z[x]$ be a monic polynomial such that $g(\alpha)=0$. Then, $g(x)=x^n+a_{n-1}x^{n-2}+\cdots +a_1x + a_0$, where $a_1,\cdots, a_{n-1}\in\Z$.
\begin{align*}
    g(\alpha)=0&\implies \left(\frac{\sqrt{2}}{3}\right)^n + a_{n-1}\left(\frac{\sqrt{2}}{3}\right)^{n-1}+\cdots + a_1\left(\frac{\sqrt{2}}{3}\right)+a_0=0\\
    &\implies (\sqrt{2})^n + a_{n-1}(\sqrt{2})^{n-1}\cdot 3+\cdot + a_1(\sqrt{2})\cdot 3^{n-1} + a_03^n = 0\\
    &\implies \displaystyle\sum_{t=0}^n (\sqrt{2})^ta_t3^{n-t}=0\\
    &\implies \underbrace{\displaystyle\sum_{t\text{ even}} (\sqrt{2})^ta_t3^{n-t}}_{\in\Z} + \underbrace{\displaystyle\sum_{t\text{ odd}} (\sqrt{2})^ta_t3^{n-t}=0}_{\in\sqrt{2}\Z\neq \Z}\\
    &\implies \overbrace{\displaystyle\sum_{t\text{ even}} (\sqrt{2})^ta_t3^{n-t}=0}^{\text{ Case 1}} \wedge \overbrace{\displaystyle \sqrt{2}\underbrace{\sum_{t\text{ odd}} (\sqrt{2})^{t-1}a_t3^{n-t}=0}_{\in\Z}}^{\text{Case 2}}\\
\end{align*}
If $n$ is even, we use Case 1 to get an extra term of $2^{\frac{n}{2}}$ and $3$ divides the remaining terms, so we reach a contradiction. If $n$ is odd, we repeat this for Case 2. Thus, no such monic $g$ exists, and so $\alpha$ is not an algebraic integer.
\end{example}
\noindent Every $\alpha\in\Q$ is an algebraic number\footnote{Take $f(x)=x-\alpha\in\Q[x]$}. Now, we consider $\Q\cap \{\text{algebraic integers}\}$.\\
\\
Let $\alpha=\frac{r}{s}\in\Q$ be an algebraic integer, where $gcd(r,s)=1$ and $s\neq 0$. There exists, then, a monic non-zero $f(x)\in\Z[x]$ such that $f(\alpha)=0$.\\
\\
Let $f(x)=x^n+a_{n-1}x^{n-1}+\cdots +a_1x+a_0$. Using the same trick as before, we multiply to get:
$$r^n=-(a_{n-1}r^{n-1}s+\cdots +a_1rs^{n-1})$$
However, this implies $s|r^n$. If $p$ is a prime dividing $s$, then $p|r^n\implies p|r\implies gcd(r,s)\geq p\implies s=1\implies \alpha=r\in\Z$.\\
\\
In conclusion, $\Z$ is the set of algebraic integers in $\Q$.
\subsection{August 29, 2022}
\begin{theorem}
Let $\alpha$ be an algebraic number. Then, there exists a unique polynomial $p(x)\in\Q[x]$ which is monic, irreducible and of lowest degree such that $p(\alpha)=0$. By definition, $p(x)$ is the \underline{minimal polynomial of $\alpha$ over $\Q$}. Furthermore, if $f(x)\in\Q[x]$ and $f(\alpha)=0$, then $p(x)|f(x)$.
\end{theorem}
\begin{proof}
Since $\alpha$ is an algebraic number, there exist a set of polynomials of the form $f(x)\in\Q[x]$ such that $f(\alpha)=0$. We choose $p(x)$ from this set to be of lowest degree. We must show that $p(x)$ is irreducible.\\
\\
Let $p(x)=a(x)b(x)$, where $a(x),b(x)\in\Q[x]$, $deg(a(x))<deg(p(x))$, and $deg(b(x))<deg(p(x))$. In other words, assume that $p(x)$ is reducible.
$$0=p(\alpha)=a(\alpha)b(\alpha)$$
Note that since $\C$ is an integral domain, we get that either $a(\alpha)=0$ or $b(\alpha)=0$. This immediately gives a contradiction, as $a(x)$ and $b(x)$ now belong to our original set of possible $f(x)'s$, and are both of lower degree than $p(x)$. Thus, $p(x)$ is irreducible. We can force this to be monic by multiplication of the inverse of the leading coefficient, since $\Q$ is a field. We have thus constructed a $p'(x)\in\Q[x]$ which is monic, irreducible, and of lowest degree. \\
\\
It remains to be shown that our monic, irreducible polynomial $p(x)$ of lowest degree is unique. Suppose $g(x)$ is another such polynomial. Since $\Q[x]$ is a Euclidean Domain, we enjoy the Division Algorithm, and so $f(x)=q(x)g(x)+r(x)$, where $q(x),r(x)\in\Q[x]$, and either $deg(r(x))<deg(g(x))$ or $r(x)=0$.
So 
$$0=p(\alpha)=q(\alpha)\underbrace{g(\alpha)}_{=0}+r(\alpha)\implies r(\alpha)=0$$
Since $g(x)$ is of minimal degree, we must then have $r(x)=0$. Thus, $p(x)=q(x)g(x)$. Since $p(x)$ and $g(x)$ are of minimal degree, they must have the same degree, which means in turn that $q(x)=c$ for some $c\in\Q$. Given now that $p(x)=cg(x)$ and that $p$ and $g$ are monic, we must have $c=1$, and so we have our result that $p(x)=g(x)$.\\
\\
Now, let $f(x)\in\Q[x]$ such that $f(\alpha)=0$. Then, $f(x)=q(x)p(x)+r(x)$ with $deg(r(x))<deg(p(x))$ or $r(x)=0$. By a similar argument as above, $p(\alpha)=0\implies r(x)=0$ by minimality of $p(x)$, so $p(x)|f(x)$.
\end{proof}
\begin{definition}
We denote the degree of $\alpha$ over $\Q$ as $deg_\Q(\alpha)=deg(p(x))$, where $p(x)$ is the minimal polynomial of $\alpha$.
\end{definition}
\begin{example}
Find the minimal polynomial of $\alpha=\sqrt{1+\sqrt{7}}$.\\
\\
Let $x=\sqrt{1+\sqrt{7}}$.
$$x^2=1+\sqrt{7}$$
$$x^2-1 = \sqrt{7}$$
$$(x^2-1)^2=7$$
$$x^4-2x^2-6=0$$
Now, let $p(x)=x^4-2x^2-6$. Then, $p(\alpha)=0$. $p$ is already monic, so we use Eisenstein's Creiterion with $p=2$. Since $2^2\not| 6$, $p(x)$ is indeed irreducible
\end{example}
\noindent As a reminder, Eisenstein's Criterion states that if $f(x)=\displaystyle\sum_{i=0}^n a_ix^i$, where $a_i\in\Z$, if there is a prime $p$ such that $p\not | a_n,$, $p|a_i$ otherwise, and $p^2\not| a_0$, then $f(x)$ is irreducible over $\Q$. Another method to keep in mind here would be the Rational Roots Theorem.
\begin{definition}
Let $E,F$ be fields, where $F\subseteq E$. We call $E$ an \underline{extension} of $F$ (or a field extension), and $F$ is denoted as the \underline{base field}. For instance, $\C$ is an extension of $\Q$. We further note that $E$ is a vector space over $F$.
\end{definition}
\noindent Recall that if $F$ is a field, and $E$ is an extension of $F$ such that $\alpha\in E$:
$$F[\alpha]=\{f(\alpha):\ f(x)\in F[x]\}\subseteq E$$
$$F(\alpha)=\{\frac{f(\alpha)}{g(\alpha)}:\ f(x),g(x)\in F[x], g(\alpha)\neq 0\}\subseteq E$$
\noindent $F[\alpha]$ is the smallest subring of $E$ containing $\alpha$, and $F(\alpha)$ is the smallest subfield of $E$ containing $\alpha$. We note that $F[\alpha]=F(\alpha)$ iff $\alpha$ is algebraic over $F$.
\begin{definition}
Let $\alpha$ be an algebraic number. Define :
$\Q[\alpha]:=\{f(\alpha):\ f(x)\in\Q[x]\}$
\end{definition}
\begin{proposition}
Let $\alpha$ be an algebraic number. $\Q[\alpha]$ is a field, which we will then denote $\Q(\alpha)$.
\end{proposition}
\begin{proof}
Let $p(x)$ be the minimal polynomial of $\alpha$ Consider $\phi_\alpha: \Q[x]\rightarrow \Q[\alpha]$, where $\phi_\alpha(f(x))=f(\alpha)$. $\phi_\alpha$ is a ring homomorphism. We note that $$ker(\phi)=\{f\in\Q[x]:\ \phi(f)=0\}=\langle p(x)\rangle$$
By the First Isomorphism Theorem, we then have that:
$$\Q[x]/\langle p(x)\rangle = \Q[\alpha]$$
Since $p(x)$ is irreducible, we have that $\langle p(x)\rangle$ is a maximal ideal, so $\Q[x]/\langle p(x)\rangle$ is a field since it is the quotient of an integral domain by a maximal ideal. Thus, $\Q[\alpha]$ is a field.
\end{proof}
\begin{definition}
A field $\Q\subseteq K\subseteq \C$ is an \underline{algebraic number field} if its dimension as a vector space over $\Q$ is finite.
\end{definition}
\noindent Suppose $F\subseteq E$ is a finite extension. We write $[E:F]=\text{dim}_F(E)$. Furthermore, every finite extension is an algebraic extension. 
\begin{definition}
$E$ is an \underline{algebraic extension} of $F$ if every element $\alpha\in E$ is algebraic over $F$. In other words, $\exists f(x)\in F[x]$ such that $f(\alpha)=0$.
\end{definition}
\noindent We note that $deg_F(\alpha)\leq [E:F]$ if $E$ is a finite extension of $F$.\\
\\
So, if $K$ is an algebraic number field, then $K=\Q(\alpha_1,\alpha_2,\cdots,\alpha_n)$ for some $n\in\N$. We note here that $\Q(\alpha,\beta)=\Q(\alpha)$. The dimension of $K$ over $\Q$ is denoted $[K:\Q]$.\\
\\
\noindent If $\alpha$ is an algebraic number, and $deg(\alpha)=n$, then $\Q(\alpha)$ is an algebraic number field, and $[\Q(\alpha):\Q]=n$, with basis $\{\alpha^i\}_{i=0}^{n-1}$.
\begin{lemma}
Let $F\subseteq \C$ be a subfield of $\C$. Let $f(x)\in F[x]$ of degree $n$ be irreducible. Then $f(x)$ has $n$ distinct roots.
\end{lemma}
\begin{proof}
Let $f(x)=\displaystyle \sum_{i=0}^n a_ix^i$. Recall the formal derivative $f'(x)=\displaystyle \sum_{i=1}^n a_i(n-i)x^{n-1}$. Assume, for the sake of contradiction, that $f(x)$ has a repeated root $\alpha\in \C$. In other words, $(x-\alpha)^2|f(x)$. Let:
$$f(x)=(x-\alpha)^2g(x)\implies p'(x)=(x-\alpha)^2g'(x)+2g(x)(x-\alpha)$$
Thus, $f'(\alpha)=0$. Let $h(x)=gcd(f(x),f'(x))\in F[x]$. Note $f'(x)\in F[x]$, and there exist $u(x),v(x)\in F[x]$ such that:
$$h(x)=u(x)f(x)+v(x)f'(x)\implies h(\alpha)=0$$
Thus, $h|f$, but $f$ is irreducible over $F$, so $h(x)=c$ or $h(x)=cf(x)$, where $c\in F\backslash \{0\}$. $h(\alpha)=0$, so we must have that $h(x)=cf(x)$. Then, $f|f'$ because $h|f'$. Thus, there are no repeated roots, so $f(x)$ has $n$ distinct roots in $\C$
\end{proof}
\begin{theorem}[Primitive Element Theorem]
If $\alpha$ and $\beta$ are algebraic numbers, then there exists an algebraic number $\theta$ such that $\Q(\alpha,\beta)=\Q(\theta)$
\end{theorem}
\begin{proof}
Let $p(x)=\displaystyle\prod_{i=1}^n x-\alpha_i$ and $q(x)=\displaystyle\prod_{i=1}^m x-\beta_i$ be the minimal polynomials of $|alpha$ and $\beta$ respectively, where $\alpha_1=\alpha$ and $\beta_1=\beta$. By the previous lemma, all these coefficients are distinct in $\C$. \\
\\
\noindent Consider for any $1\leq i\leq n$, $2\leq j\leq m$:
\begin{equation}
\alpha_i+\lambda\beta_j = \alpha+\lambda \beta
\end{equation}
This implies that $\lambda_{ij} = \frac{\alpha_i-\alpha}{\beta_j-\beta}$. Thus, Equation $(1)$ holds for exactly one value of $\lambda\in\C$ (for a fixed $i,j$) and at most one $\lambda\in\Q$. \\
\\
Now, choose $0\neq c\in\Q$ such that $\alpha_i+c\beta_j\neq \alpha + c\beta$, for every $1\leq i\leq n $, and every $2\leq j \leq m$. Such a $c$ always exists because there are only finitely many extensions. This choice is equivalent to choosing $0\neq c\in\lambda$ such that $c\neq \lambda_{ij}$ for all $i,j$. \\
\\
Now, let $\theta=\alpha+C\beta$. We will now show that $\Q(\alpha,\beta)=\Q(\theta)$. Note that $\theta=\alpha+c\beta\implies \theta\Q(\alpha,\beta)\implies \Q(\theta)\subseteq \Q(\alpha,\beta)$. It remains to show that backwards inclusion. \\
\\
If $\beta\in \Q(\theta)$, then $\alpha=\theta-c\beta\in\Q(\theta)$, so it suffices to show that $\beta\in\Q(\theta)$. Let $r(x)+p(\theta-cx)\in \Q(\theta)[x]$, and $r(\beta)=p(\theta-c\beta)=p(\alpha)=0$. So $\beta$ is a common root of $r(x)$ and $q(x)$. Let $t$ be another common root of $r(x)$ and $q(x)$. Then, $t\in\{\beta_j\}_{j=2}^m$. Now, for $2\leq j\leq m$, $r(\beta_j)=p(\theta-c\beta_k)$\\ \\
\noindent So, if $0=r(\beta_j)$, then $0=p(\theta-c\beta_k)\implies \theta-c\beta_j\in\{\alpha\}_{i=1}^n$. Thus, $\beta$ is the only common root of $r(x)$ and $p(x)$. \\
\\
Let $h(x)$ be the minimum polynomial of $\beta$ over $\Q(\theta)$. This implies that $h(x)|r(x)$ and $h(x)|q(x)$, so $\beta$ is the only root of $h(x)$ in $\C$, so $deg(h(x))=1$, and in particular, $h(x)=x-\beta\in \Q(\theta)[x]$. This means we must have that $\beta\in\Q(\theta)$, and so we are done! 
\end{proof}
\noindent By induction, $\Q(\alpha_1,\cdots, \alpha_n\}=\Q(\theta)$. Thus, all algebraic number fields can be represented as $\Q(\theta)$ for some algebraic number $\theta$.
\subsection{September 2, 2022}
Recall, from last time, that all algebraic number fields can be represented by $\Q(\alpha)$ for some algebraic number $\alpha$.
\begin{theorem}
Every algebraic number $\theta$ is of the form $\frac{\alpha}{c}$ where $\alpha$ is an algebraic integer and $c\in \Q$. 
\end{theorem}
\begin{proof}
Let $f(x)$ be the minimum polynomial of $\theta$, $deg(f)=n$. The coefficients of $f$ are rational. Let $c$ be the lowest common multiple of all the denominators of the $a_i$'s. Now, consider:
$$g(x)=x^n+ca_{n-1}x^{n-1}+c^2a_{n-2}x^{n-2}+\cdots c^na_0$$
Note the general term above is $c^ta_{n-t}x^{n-t}$. Now:
$$g(c\theta) = c^n(\theta^n+a_{n-1}\theta^{n-1}\cdots ) = f(\theta)=0$$
Note that $g(x)\in\Z[x]\implies c\theta$ is an algebraic integer, so let $\alpha=c\theta$. Then, $\theta=\frac{\alpha}{c}$ as desired.
\end{proof}
\begin{cor}
Every algebraic number field can be represented by $\Q(\alpha)$ for some algebraic integer $\alpha$
\end{cor}
\begin{proof}
We know every algebraic number field can be represented by $\Q(\theta)$, where $\theta$ is an algebraic number. But $\theta=\frac{\alpha}{c}$. where $c\in\Z$ and $\alpha$ is an algebraic integer, so $\Q(\theta)=\Q(\frac{\alpha}{c})=\Q(\alpha)$.
\end{proof}
\begin{example}
Find the value of $\theta$ such that:
$$\Q(\sqrt{2},\sqrt[3]{6}) = \Q(\theta)$$
The minimum polynomial of $\sqrt{2}$ is $f(x)=x^2-2=(x-\sqrt{2})(x+\sqrt{2})$. The minimum polynomial of $\sqrt[3]{5}$ is $g(x)=x^3-5=(x-\sqrt[3]{5})(x-w\sqrt[3]{5})(x-w^2\sqrt[3]{5})$, where $w=-\frac{1}{2}+\frac{\sqrt{3}}{2}i$. Choose $c$ such that $\alpha_i+c\beta_j\neq \alpha+c\beta$ for any $i=1,2$, $j=2,3$. Choose $c=1$, and this works. So take $\theta=\alpha+c\beta = \alpha+\beta =\sqrt{2}+\sqrt[3]{5}$.\\
\\
The minimum polynomial of $\sqrt{2}+\sqrt[3]{5}$ can be left as an exercise, but turns out to be $f(x)=x^6-6x^2-10x^3+12x^2-60x+16$.
\end{example}
\begin{definition}
Now, let $\alpha$ be an algebraic number with minimum polynomial of degree $n$. Then $$p(x)=\displaystyle\prod_{i=1}^nx-\alpha_i \in \Q[x]$$ where by the Lemma, all $\alpha_i$'s are distinct complex numbers. $\alpha=\alpha_1, \alpha_2,\alpha_3,\cdots, \alpha_n$ are called the \underline{conjugates} of $\alpha$.
\end{definition}
Then, we have $n$ field isomorphisms (embeddings):
$$\sigma_i: \Q(\alpha)\rightarrow \Q(\alpha_i)$$
with $\alpha\mapsto \alpha_i$.
Note: These conjugate fields are independent of choice of $\alpha$. \\
\\
If $K=\Q(\alpha)=\Q(\beta)$, then $\exists$ $c_i\in\Q$ such that
$$\beta = c_0+c_1\alpha + c_2\alpha^2+\cdots + c_{n-1}\alpha^{n-1}$$
For $t=1,2,\cdots, n$, set
$$\beta_t=c_0+c_1\alpha_1+c_2\alpha_2^2+\cdots + c_{n-1}\alpha_t^{n-1}=\sigma_i(\beta)$$
Then the $\beta_i$'s are the conjugates of $\beta$ and $\Q(\alpha_t)=\Q(\beta_t)$, for every $t=1,2,\cdots, n$.
\begin{definition}
Let $S_n$ denote the symmetric group on $n$ letters. For any $\sigma\in S_n$ and $f\in\mathbb{F}[x_1,x_2,\cdots, x_n]$, where $\mathbb{F}$ is a field, define
$$f^\sigma(x_1,x_2,\cdots, x_n)=f(x_{\sigma(1)},\cdots, x_{\sigma(n)})$$
A polynomial $f$ is symmetric if $f^\sigma=f$, for every $\sigma\in S_n$
\end{definition}
\begin{example}
$f(x_1,x_2,x_3)=x_1+x_2+x_3$ is symmetric. 
\end{example}
\begin{example}
$f(x_1,x_2,x_3)=x_1+x_2x_3$. Then, if $\sigma = (1\ 2\ 3)$, $f^\sigma=x_2+x_3x_1$, so $f$ is not symmetric.
\end{example}
\begin{definition}
The \underline{elementary symmetric polynomials} are defined as:

$$e_0(x_1,\cdots, x_n) = 1$$
$$e_1(x_1,\cdots, x_n)=x_1+x_2+\cdots + x_n$$
$$e_2(x_1,\cdots, x_n) = \displaystyle\sum_{1\leq i\leq j\leq n}x_ix_j$$
$$e_n(x_1,\cdots, x_n)=x_1x_2\cdots x_n$$
\end{definition}
\begin{theorem}[Fundamental Theorem of Symmetric Polynomials]
Every symmetric polynomial can be written uniquely as a polynomial expression (not necessarily symmetric) in the elementary symmetric polynomials. In other words, if $f(x_1,\cdots, x_n)\in\mathbb{F}[x_1,\cdots, x_n]$, is a symmetric polynomial, then there exists a $g$ such that $f(x_1,\cdots, x_n)=g(e_1(x_1,\cdots, x_n),\cdots, e_n(x_1,\cdots, x_n))$. $\mathbb{F}$ could just be a commutative ring that is not a field.
\end{theorem}
\newpage
\subsection{September 7, 2022}
\begin{example}
Let $f(x_1,x_2)=x_1^3+x_2^3-7$ be symmetric. Let $g(x_1,x_2)=x_1^3-3x_1x_2-7$ be not symmetric. But then, we see that:
\begin{align*}
    f(x_1,x_2)&=g(e_1(x_1,x_2)+e_2(x_1,x_2))\\
    &=q(x_1,x_2)^3-3e_1(x_1,x_2)e_2(x_1x_2)-7
\end{align*}
\end{example}
\begin{proposition}
Suppose $p(x)\in\Q[x]$ (monic) has roots $\alpha_1,\cdots,\alpha_n$, and any symmetric polynomial $f(x_1,\cdots, x_n))\in\Q[x_1,\cdots,x_n]$, then $f(\alpha_1,\cdots,\alpha_n)\in\Q$
\end{proposition}
\begin{proof}
$p(x)=\displaystyle\prod_{i=1}^n(x-\alpha_i)\in\Q[x]$, so $e_t(\alpha_1,\cdots, \alpha_n)\in\Q$, for every $1\leq t\leq n$. By the Fundamental Theorem of Symmetric Polynomials, there exists $g(x_1,\cdots, x_n)\in\Q[x_1,\cdots, x_n]$ such that $f(x_1,\cdots, x_n)=g(e_1(x_1,\cdots, x_n),\cdots, e_n(x_1,\cdots, x_n))$. So:
$$f(\alpha_1,\cdots, \alpha_n)=g(e_1(x_1,\cdots, x_n),\cdots, e_n(x_1,\cdots, x_n))\in\Q$$
\end{proof}
\begin{theorem}
Embeddings are independent of choice of $\alpha$. In other words, let $K=\Q(\alpha)$ be an algebraic number field, and $p(x)=\displaystyle\prod_{i=1}^n(x-\alpha_i)\in\Q$ be a minimum polynomial of $\alpha$ with $\alpha=\alpha_1$. The embeddings are $\alpha\rightarrow \alpha_i$. If $K=\Q(\beta)$ also, $\beta=c_0+c_1\alpha+c_2\alpha^2+\cdots +c_{n-1}\alpha^{n-1}$ because these powers from a basis for $K=\Q(\alpha)$. Let $\beta_i=\sigma_i(\beta)$ for $1\leq i\leq n$. Then, $\beta=\beta_1,\beta_2,\cdots, \beta_n$ are the conjugates of $\beta$ and $\Q(\alpha_i)=\Q(\beta_i)$ for every $1\leq i\leq n$.
\end{theorem}
\begin{proof}
Let
\begin{align*}
    f(x)&=\displaystyle\prod_{i=1}^n(x-\beta_i)\\
    &=\prod_{i=1}^n(x-\sigma_i(\beta))\\
    &=\prod_{i=1}^n(x-(c_0+c_1\alpha_i+c_2\alpha_i^2+\cdots +c_{n-1}\alpha_i^{n-1}))\\
    &\in \underbrace{\Q(\alpha_1,\alpha_2,\cdots,\alpha_n)}_{K_1}[x]
\end{align*}
So $\Q\subseteq K\subseteq K_1\subseteq \C$. Now:
\begin{align*}
    p(x)&=\displaystyle\prod_{i=1}^n(x-\beta_i)\\
    &=x^n-e_1(\beta_1,\cdots, \beta_n)x^{n-1}+\cdots + (-1)^ne_n(\beta_1,\cdots, \beta_n)
\end{align*}
Each $e_i(\beta_1,\cdots, \beta_n)$ will be symmetric in the $\alpha$'s. Thus, by the fundamental theorem, there exists $g(x_1,\cdots, x_n)\in\Q[x]$ such that $e_t(\beta_1,\cdots, \beta_n)= g_t(\alpha_1,\cdots, \alpha_n)$. From Proposition 3, we get that $g_t\in\Q$, so each $e_t\in\Q$, so $f(x)\in\Q[x]$.\\
\\
S0 $f(x)=\displaystyle\prod_{i=1}^n (x-\beta_i)\in \Q[x]$. $f(\beta)=0$, so if $h(x)$ is the minimal polynomial for $\beta$, then $h|f$. But $deg(f)=n$ and $[\Q(\beta):\Q]=[\Q(\alpha):\Q]=n$, so $deg(h)=n$. Since $h$ and $f$ are monic of the same degree, we must have that $h=f$, so $f(x)$ is the minimum polynomial of $\beta$. But the conjugate fields are $\Q(\beta_i)\subseteq \Q(\alpha_i)$, and both are of degree $n$, so finally we get that $\Q(\beta_i)=\Q(\alpha_i)$.
\end{proof}
\newpage
\subsection{September 9, 2022}
\begin{definition}
Let $\mathcal{O}\subseteq \C$ be the set of all algebraic integers
\end{definition}
\begin{theorem}
The following are equivalent:
\begin{enumerate}[(a)]
    \item $\alpha$ is an algebraic integer
    \item The minimum polynomial of $\alpha$ is in $\Z[x]$.
    \item $\Z[\alpha]$ is a finitely generated $\Z$-module. 
    \item There exists a finitely generated $\Z$-submodule of $\C$, $M\neq \{0\}$ such that $\alpha M\subseteq M$.
\end{enumerate}
\end{theorem}
\begin{proof}
$(a)\implies (b)$:\\
\\
Since $\alpha$ is an algebraic integer, there exists $f(x)\in\Z[x]$ monic such that $f(\alpha)=0$. Choose $p(x)$ of minimum degree from such $f(x)$'s. Then, $p(X)$ is irreducible over $\Z$\footnote{If not, there exists a lower degree polynomial of which $\alpha$ is a root, thus providing a contradiction}. Thus, $p(x)$ is irredcible over $\Q$ by Gauss' Lemma, and so $p(x)$ is the minimal polynomial of $\alpha$ over $\Q$. \\
\\
$(b)\implies (c)$:
$$\Z[\alpha]=\{f(x):\ f(x)\in\Z[x]\}$$
Now, $\Z[\alpha]$ is generated by $\{1,\alpha,\cdots,\alpha^{n-1}\}$, where $deg_\Q\alpha = n$. Let $f(x)$ be the polynomial of $\alpha$. Then:
\begin{align*}
    0 &= f(\alpha)=\displaystyle\sum_{i=0}^n a_i\alpha^i\\
    &\implies \alpha^n\in \text{Span}(\{1,\alpha,\cdots, \alpha^{n-1}\})
\end{align*}
By induction, $\alpha^N\in \text{Span}(\{1,\alpha,\cdots, \alpha^{n-1}\})$, for every $N\geq n$. So $\Z[\alpha]\subseteq \text{Span}_\Z)\{1,\alpha,\cdots, \alpha^{n-1}\})$. The reverse inclusion is obviously true, so we get equality, and thus we have as desired that $\Z[\alpha]$ is a finitely generated $\Z$-module.\\
\\
$(c)\implies (d)$:\\
\\
Let $M=\Z[\alpha]$. Then, trivially, $\alpha M\subseteq M$.\\
\\
$(d)\implies (a):$\\
\\
Let $M$ be a finitely generated $\Z$-submodule of $\C$ such that $\alpha M\subseteq M$. Let:
$$M=\Z x_1+\Z x_2+\cdots + \Z x_n$$
In other words, the $x_1,\cdots x_n$'s are the generators. Each $x_i\in M$, so $\alpha x_i\in M$ by $(c)$. Now, let:
$$\alpha x_i = \displaystyle\sum_{j=1}^n c_yx_j$$
for $1\leq j\leq n$, $C_y\in \Z$. Let $C=(C_y)$ as an $n\times n$ matrix. So:
\begin{align*}
    &(C-\alpha I)\begin{pmatrix} x_1\\ \vdots \\ x_n\end{pmatrix}=0\\
    &\implies \text{det}(C-\alpha I)=0
\end{align*}
The above follows because the vector $(x_1,\cdots, x_n)^top$ is non-zero. Now, let $f(x)=(-1)^n \text{det}(C-\lambda I)$. We then get that $f(\alpha)=0$. Also, $f(x)$ is monic in $\Z[x]$ because $c_{ij}\in\Z$. Thus, $\alpha$ is an algebraic integer.
\end{proof}
\begin{proposition}
$\mathcal{O}$ is a ring.
\end{proposition}
\begin{proof}
Let $\alpha,\beta$ be algebraic integers of degrees $n$ and $m$. We want to show that $\alpha\pm \beta$ and $\alpha\beta$ are also algebraic integers, essentially amounting to the Subring Test. \\
\\
So, $1,\alpha,\cdots, \alpha^{n-1}$ generate $\Z[\alpha]$ and $1,\beta,\cdots, \beta^{n-1}$ generate $\Z[\beta]$ as $\Z$-modules. Then, $\alpha^i\beta^j$ span $\Z[\alpha,\beta]$, for $1\leq i\leq n$, $1\leq j\leq m$. Thus, $M=\Z[\alpha,\beta]$ is a finitely generated submodule of $\C$, non-zero , and noting that $(\alpha\pm \beta)M\subseteq M$, and $\alpha\beta M\subseteq M$, and thus $\alpha\pm \beta$ and $\alpha\beta$ are algebraic integers.
\end{proof}
\begin{definition}
Let $K$ be an algebraic number field. Then define $\mathcal{O}_K=\mathcal{O}\cap K$ to be the set of all algebraic integers in $K$.
\end{definition}
\begin{cor}
$\mathcal{O}_K$ is a ring.
\end{cor}
\begin{proof}
This is the intersection of two rings.
\end{proof}
\noindent We have seen that $\mathcal{O}_\Q = \Z$. Also, $\mathcal{O}_{\Q(i)}=\Z[i]$.
\begin{definition}
Let $K$ be an algebraic number field of degree $n$. Let $\omega_1,\omega_2,\cdots, \omega_n\in K$. Let $\sigma_i$ for $1\leq i\leq n$ denote the $n$ distinct embeddings of $K$. For $j=1,\cdots, n$, let:
$$\omega_j^{(i)}=\sigma_i(\omega_j)$$
Then, the \underline{discriminant} of $\omega_1,\cdots, \omega_n$ is denoted $D(\omega_1,\cdots, \omega_n)$ (or sometimes $\Delta(\omega_1,\cdots, \omega_n)$, and is computed as:
$$D(\omega_1,\cdots, \omega_n)=(\text{det}(\sigma_i(\omega_j))_{ij})^2$$
\end{definition}
\subsection{September 12, 2022}
Let $\alpha_1,\cdots, \alpha_n$ and $\beta_1,\cdots , \beta_n$ be two bases for $K$. Let:
$$\beta_j=\displaystyle\sum_{i=1}^nc_{ij}\alpha_i$$
for some $c_{ij}\in\Q$. Let $C=(c_{ij})_{ij}$, for $1\leq i,j\leq n$. Consider:
\begin{align*}
    (\sigma_i(\beta_j))_{ij}&=(\sigma_i\left(\sum_{k=1}^nc_{kj}\alpha_k\right))_{ij}\\
    &=\left(\sum_{k=1}^nc_{kj}\sigma_i(\alpha_k)\right)_{ij}\\
    &=\left(\sum_{k=1}^n\sigma_i(\alpha_k)c_{kj}\right)_{ij}\\
    &=(\sigma_i(\alpha_j))_{ij}C
\end{align*}
\begin{align*}
    D(\beta_1,\cdots , \beta_n)&=(\text{det}(\sigma_i(\beta_j)_{ij}))^2\\
    &=(\text{det}(\sigma_i(\alpha_j)_{ij}))^2(\text{det} C)^2\\
    &=D(\alpha_1,\cdots ,\alpha_n)(\text{det} C)^2
\end{align*}
We define, for $\alpha\in K$:
$$D(\alpha)=D(1,\alpha, \cdots ,\alpha^{n-1})$$
So:
$$D(\alpha)=\underbrace{\left[ \prod_{1\leq i\leq j\leq n} (\alpha^j-\alpha^i)\right]^2}_{\text{Vandermonde Determinant}}$$
\begin{example}
Suppose $K=\Q(\sqrt{d})$, where $d$ is square-free. The minimum polynomial of $\sqrt{d}$ over $\Q$ is:
$$p(x)=x^2-d = (x-\sqrt{d})(x+\sqrt{d})$$
Considering the embedding $\sigma_1: Id$, $\sigma_2: \sqrt{d}\mapsto -\sqrt{d}$, we get:
$$D(\sqrt{d})=D(1,\sqrt{d})=\left|\begin{array}{cc}
    1 & \sqrt{d} \\
    1 & -\sqrt{d}
\end{array}\right|^2=(-\sqrt{d}-\sqrt{d})^2=4d$$
Also, $n=2$, so:
$$D(\alpha)=\prod_{1\leq i\leq j\leq n}(\alpha^j-\alpha_i)^2 = (-\sqrt{d}-\sqrt{d})^2=4d$$
\end{example}
\begin{example}
Let $K=\Q\sqrt[3]{2}$. We get a minimum polynomial using the primitive cube roots of unity where $w=-\frac{1}{2}+\frac{\sqrt{3}}{2}i$.
\begin{align*}
    D(\alpha)&=\prod_{1\leq i\leq j\leq n}(\alpha^j-\alpha^i)^2\\
    &=[(w^2\sqrt[3]{2}-w\sqrt[3]{2})(w^2\sqrt[3]{2}-\sqrt[3]{2})(w\sqrt[3]{2}-\sqrt[3]{2})]^2\\
    &=-108
\end{align*}
\end{example}
\noindent Note that $D(w_1,\cdots ,w_n)=(\text{det}((\omega_j^{(i)})_{ij}))^2$ is a symmetric function of the $w$'s.
\begin{theorem}
Let $K$ be an algebraic number field, and $w_i\in K$.
\begin{enumerate}[(a)]
    \item $D(w_1,\cdots, w_n)\in\Q$.
    \item If $w_1,\cdots, w_n\in \mathcal{O}_K$, then $D(w_1,\cdots ,w_n)\in\Z$
    \item $D(w_1,\cdots, w_n)\neq 0$ if and only if $w_1,\cdots, w_n$ are linearly independent.
\end{enumerate}
\end{theorem}
\begin{proof}
\textbf{(a):}\\
\\
$K=\Q(\theta)$ for some algebraic number $\theta$. So a basis for $K$ is $1,\theta, \cdots ,\theta^{n-1}$ where $\text{deg}_\Q K=n$. So:
$$w_j=c_{0j}+c_{1j}\theta+ \cdots + c_{n-1,j}\theta^{n-1}$$
\begin{align*}
    D(w_1,\cdots ,w_n)&=(\text{det}((w_k^{(i)})_{ij}))^2\\
    &=(\text{det}\left(\sum_{t=0}^{n-1}c_{ij}\theta^t\right))^2
\end{align*}
This is symmetric in $\theta_1,\theta_2,\cdots ,\theta_n$. Since permuting $\theta$'s just permutes the rows of the matrix, let:
\begin{align*}
    f(x_0,\cdots ,x_{n-1})=(\text{det}\left(\sum_{t=0}^{n-1}c_{ij}x^t\right))^2\\
    &\in \Q[x_0,\cdots ,x_{n-1}]\\
\end{align*}
By Proposition 3, we get that $D(w_1,\cdots , w_n)=f(\theta_1,\cdots ,\theta_n)\in\Q$.\\
\\
\textbf{(b):}\\
\\
By part $(a)$, $D(w_1,\cdots , w_n)\in\Q$, but if $w_1,\cdots , w_n\in\mathcal{O}$, then we have by definition that $D(w_1,\cdots ,w_n)\in\mathcal{O}$, since $\mathcal{O}$ is a ring. Thus, $D(w_1,\cdots ,w_n)=\Q\cap \mathcal{O} = \Z$\\
\\
\textbf{(c):}\\
\\
For the forward direction, we prove this by the contrapositive. In other words, if $w_1,\cdots ,w_n$ are not linearly independent, then $D(w_1,\cdots ,w_n)=0$. Let $w_1,\cdots ,w_n$ be linearly dependent. Then, $\exists c_1,\cdots ,c_n$ not all zero, from $\Q$ such that $\displaystyle\sum_{i=1}^nc_iw_i=0$. Applying the embeddings of $K$, we get:
$$c_iw_i^{(i)}+c_2w_2^{(i)}+\cdots +c_nw_n^{(i)}=0$$
We know the induced matrix has a non-zero solution for $\vec{c}-(c_1,\cdots ,c_n)^\top$ in $\Z$. Thus, the induced matrix is NOT invertible, so its determinant is 0. Thus, $D(w_1,\cdots ,w_n)=0^2=0$, so we have the proof of the forward direction by the contrapositive.\\
\\
Now, for the converse, let $w_1,\cdots ,w_n$ be linearly independent. Then, these form a basis for $K$ over $\Q$. Let $K=\Q(\theta)$. Then, $1,\theta,\cdots, \theta^{n-1}$ is also a basis for $K$. Thus, there exists a change of basis matrix $C\neq 0$ such that:
$$D(1,\theta,\cdots, \theta^{n-1})=(\text{det}(C))^2D(w_1,\cdots, w_n)$$
However, we are more familiar with the left side by the Vandermonde Determinant, where:
$$D(1,\theta,\cdots, \theta^{n-1})=\displaystyle\prod_{1\leq i\leq j\leq n}(\theta^{(j)}-\theta^{(i)})^2$$
Here, $\theta^{(i)}=\sigma_u(\theta)$ is the $i^{th}$ conjugate of $\theta$, and all the conjugates are distinct. Thus, we get that this discriminant is nonzero. Thus, $D(w_1,\cdots, w_n)\neq 0$
\end{proof}
\newpage
\section{Integral Bases}
\subsection{September 14, 2022}
\begin{definition}
Let $K$ be an algebraic number field. A $\Z$-basis for $\mathcal{O}_K$ is called an \underline{integral basis for $K$} (but really for $\mathcal{O}_K$.
\end{definition}
\begin{proposition}
Every $\Z$-basis for $\mathcal{O}_K$ is a $\Q$ basis for $K$. 
\end{proposition}
\begin{proof}
Let $w_1,\cdots w_t$ be a $\Z$ basis for $\mathcal{O}_K$. Let $\theta\in K$ be an algebraic number. Then, $\theta=\frac{\alpha}{m}$, where $\alpha\in\mathcal{O}_K$ is an algebraic integer, and $m\in\Z$. This implies that, uniquely, $\alpha=\displaystyle\sum_{i=1}^tc_iw_i$, for $c_i\in\Z$. Then, $\theta\in Spqn_\Q(w_1,\cdots, w_t)\implies K\subseteq Span_\Q(w_1,\cdots, w_t)\implies Span_\Q(w_1,\cdots, w_t)=K$. Suppose $w_1,\cdots, w_t$ was linearly dependent over $\Q$. Then, there exists $q_1,\cdots, q_t$ not all zero such that the finite linear combination sums to 0. Then, we can multiply by $n\in\Z$ to get this linear combination to sum to 0, which then, by the linear independence over $\Z$, gives rise to an obvious contradiction which then shows us that $w_1,\cdots, w_t$ is linearly independent over $\Q$, and so must be a $\Q$ basis for $K$. Combined with a counting argument where the sizes of these two bases are the same, we get our desired result.
\end{proof}
\noindent Note: Not all bases of $K$ will be integral bases.
\begin{example}
$K=\Q(\sqrt{5})$ has a basis $1,\sqrt{5}$. The minimum polynomial is $x^2-5$ which has degree 2. However, the above basis is NOT an integral basis. $\frac{1+\sqrt{5}}{2}\in\air$ has minimum polynomial $x^2-x-1\in\Z[x]$, but this is NOT in the span of the basis over $\Z$. However, $\{1,\frac{1+\sqrt{5}}{2}\}$ is an integral basis, where $\air =\Z\left[\frac{1+\sqrt{5}}{2}\right]$.
\end{example}
\noindent In general, if $K=\Q(\alpha)$, a basis $\{1,\alpha,\cdots, \alpha^{n-1}\}$ is not necessarily an integral basis. $\Z[\alpha]\subseteq \air$, but $\air\not\subseteq \Z[x]$ in general.
\begin{theorem}
Every number field $K$ has an integral basis.
\end{theorem}
\begin{proof}
Let $K=\Q(\alpha)$, where $\alpha$ is an algebraic integer. $\{1,\alpha,\cdots, \alpha^{n-1}\}$ is a basis for $K/\Q$, $deg_\Q K=n$. Now, $D(\alpha)>0 \in\Z$, because $\{1,\alpha,\cdots, \alpha^{n-1}\}$ is a linearly independent set in $\air$. Now, we select a basis $\{w_1,\cdots, w_n\}$ of $K$ from $\air$, such that $D(w_1,\cdots, w_n)$ is minimal. We'll now show that this is an integral basis. \\
\\
Suppose otherwise. There then exists $w\in\air$ such that $w=\displaystyle\sum_{i=1}^n a_iw_i$ such that $a_i\in\Q$ and there is at least one $a_i\in \Q\setminus \Z$. Without loss of generality, assume $a_1\in \Q\setminus \Z$. We write $a_1=a+r$, where $a\in\Z$ and $r\in \Q\setminus \Z$, $0<r<1$. Define $\phi_1=w-aw_1$, $\phi_i=w_i$ otherwise. Then, by construction, $\{\phi_1,\phi_2,\cdots, \phi_n\}$ is also a $\Q$-basis for $K$,where $\phi_i\in\air$. Now, we consider a change of basis matrix $C$ such that $\Phi=CW$. Then:
$$C=\left|\begin{array}{ccccc}
   a_1-a  & a_2 & a_3 &\cdots &a_n \\
   0  & 1 & 0 &\cdots &0\\
   0 & 0 & 1 & \cdots &0\\
   \vdots & \vdots &\vdots &\ddots &\vdots\\
   0 & 0 & 0 & \cdots & 1
\end{array}\right|$$
Thus, $\text{det}(C)=a_1-a=r$, and $D(\phi)=r^2D(w)$, where $0<r<1$, but this leads to a contradiction, as this gives $D(\phi)<D(w)$, a contradiction to $D(w)$ being minimal.
\end{proof}
\begin{theorem}
Let $\alpha_1,\cdots, \alpha_n$ be a basis for $K$ over $\Q$. If $D(\alpha_1,\cdots, \alpha_n)$ is square free, then $\{\alpha_1,\cdots, \alpha_n\}$ is an integral basis.
\end{theorem}
\begin{proof}
Let $\beta_1,\cdots \beta_n$ be an integral basis for $K$. Then there exist $c_{ij}\in\Z$ such that $\alpha_i=\displaystyle\sum_{j=1}^n c_{ij}\beta_j$ for all $1\leq \leq n$ because $\alpha_i\in\air$. Now, Let $C=(c_{ij})$. So:
$$\underbrace{D(\alpha_1,\cdots, \alpha_n)}_{\in\N}=\underbrace{(\text{det}(C))^2}_{\in\Z}\underbrace{D(\beta_1,\cdots, \beta_n)}_{\in\Z}$$
We note that since the left side is square free, $\text{det}(C)=\pm 1$, so $C$ is invertible over $\Z.$ This in turn implies that $C$ is an integral change of basis matrix.
\end{proof}
\begin{remark}
The converse is false.
\end{remark}
\begin{example}
If $K=\Q(i)$, then $\air = \Z[i]$. We have a basis $\{1,i\}$ with nontrivial embedding $i\mapsto -i$, and minimal polynomial $x^2+1=(x-i)(x-(-i))$. $D(1,i)=4$ is not square free, but $\{1,i\}$ is an integral basis.
\end{example}

\begin{remark}
For two integral bases $\{\alpha_1,\cdots,\alpha_n\}$ and $\{\beta_1,\cdots,\beta_n\}$, we always get that their discriminants are the same. In other words, the discriminant of an integral basis is independent of choice of basis. 
\end{remark}
\begin{definition}
The \underline{discriminant of $K$}, denoted $D_K,\Delta_K$ is exactly the discriminant mentioned in Remark 2.
\end{definition}
\begin{example}
$K=\Q(\sqrt{5})$ has basis $\{1,\sqrt{5}\}$, but we saw that this is not an integral basis in Example 9. So what IS an integral basis here? We have from Example 9, without proof, that $\left\{1,\frac{1+\sqrt{5}}{2}\right\}$ is an integral basis.
\begin{align*}
    D\left(1,\frac{1+\sqrt{5}}{2}\right)&=\left|\begin{array}{cc}
        1 & \frac{1+\sqrt{5}}{2} \\
        1 & \frac{1-\sqrt{5}}{2}
    \end{array}\right|^2\\
    &=\left(\frac{1-\sqrt{5}}{2}-\frac{1+\sqrt{5}}{2}\right)^2=5
\end{align*}
$5$ is square free, so this is an integral basis.
\end{example}
\begin{definition}
Let $K$ be a number field with $\Q$ basis $\{w_1,\cdots, w_n\}$. Let $\alpha\in K$ and let $\alpha w_i=\displaystyle\sum_{j=1}^na_{ij}w_j$, for all $1\leq i\leq n$, and $a_{ij}\in\Q$. Let $A=(a_{ij})$ be an $n\times n$ matrix.
We define the \underline{trace of $\alpha$}, $Tr_K(\alpha)$ such that $Tr_K(\alpha)=Tr(A)$, and the \underline{norm of $\alpha$}, $N_K(\alpha)$ as $N_K(\alpha)=\text{det}(A)$.
\end{definition}
\begin{example}
Consider $K=\Q(\sqrt{d})$, where $d$ is square-free integer. This has a basis $\{1,\sqrt{d}\}$. Let $\alpha=a+b\sqrt{d}$, $a,b\in\Q$.
$$\alpha w_1=(a+b\sqrt{d})=a\cdot 1 + b\cdot \sqrt{d}$$
$$\alpha w_2=(a+b\sqrt{d})(\sqrt{d})=(bd)\cdot 1 + a\cdot\sqrt{d}$$
So:
$$A_\alpha = \begin{pmatrix}a & b\\ bd & a\end{pmatrix}$$
This gives us that $Tr(\alpha)=2a$, $N(\alpha)=a^2-db^2$.
\end{example}
\newpage
\begin{lemma}
Let $K$ be a number field, If $\alpha\in\air$, then $Tr_K(\alpha),N_K(\alpha)\in\Z$
\end{lemma}
\begin{proof}
Let $w_1,\cdots, w_n$ be a basis $\Q$-basis for $K$. Let $\alpha w_i=\displaystyle\sum_{j=1}^na_{ij}w_j$, for every $1\leq i\leq n$, $a_{ij}\in\Q$. Then, let $A=(a_{ij})$.
\\
\\
Let $\sigma_1,\cdots, \sigma_n$ be the embeddings of $K$. Take $\sigma_k$ of $\alpha w_i$.
\begin{align*}
    &\sigma_k(\alpha w_i) = \sigma_k\left(\sum_{j=1}^n a_{ij} w_j\right)\\
    \implies &\sigma_k(\alpha)\sigma_k(w_i) =\sum_{j=1}^n a_{ij}\sigma_k(w_j)\\
    \implies &\sum_{j=1}^n \delta_{jk}\sigma_j(\alpha)\sigma_j(w_i)=\sum_{j=1}^n a_{ij}\sigma_k(w_j)\\
\end{align*}
Now, define:
$$A_0 = (\delta_{ij}\sigma_i(\alpha))_{ij}$$
$$M= (\sigma_j(w_i))_{ij}$$
We note that $0\neq D(w_1,\cdots, w_n) = \text{det}(M^T)^2 = \text{det}(M)^2 \implies M$ is invertible.\\
\\
$$AM=\left(\sum_{k=1}^n a_{ik}\sigma_j(w_k)\right)_{ij}$$
$$MA_0 = \left(\sum_{k=1}^n\sigma_k(w_i)\delta_{jk}\sigma_k(\alpha)\right)_{ij}$$
Thus, by the above, we get that $AM=MA_0\implies A_0 = M^{-1}AM\implies \text{det}A_0=\text{det} A$ (and their traces are equal). Thus, the trace is the sum of the $\sigma_i(\alpha)'s$ and the norm ia the product. This gives us that the trace and norm of $\alpha$ are independent of choice of basis.
\end{proof}
\newpage
\begin{proposition}
Let $K=\Q(\alpha)$, $\alpha\in\mathcal{O}$. Let $p(x)$ be the minimum polynomial of $\alpha$ of degree $n$. Then, $D(\alpha)=(-1)^{{{n}\choose{2}}} N_K(p'(\alpha))$
\end{proposition}
\begin{proof}
By the Vandermonde Determinant, $D(\alpha)=\displaystyle\prod_{1\leq i\leq j\leq n}(\alpha^{(j)}-\alpha^{(i)})^2$. On the left, we have:
$$p(x)=\prod_{i=1}^n (x-\sigma_i(\alpha))$$
$$p'(x)=\sum_{j=1}^n \prod_{i\neq j} (x-\sigma_i(\alpha)) =\sum_{j=1}^n \prod_{i\neq j} (x-\alpha^{(i)})$$
$$\implies p'(\alpha^{(k)}) = \sum_{j=1}^n \prod_{i\neq j} (\alpha^{(k)}-\alpha^{(i)})=\prod_{i\neq k}(\alpha^{(k)}-\alpha^{(i)})$$
\begin{align*}
    N(p'(\alpha))&=\prod_{j=1}^n \sigma_j(p'(\alpha))\\
    &=\prod_{j=1}^n p'(\alpha^{(j)})\\
    &=\prod_{j=1}^n\prod_{i\neq j} (\alpha^{(j)}-\alpha^{(i)})\\
    &=\prod_{1\leq i < j\leq n} (-1)^s(\alpha^{(j)}-\alpha^{(i)})^2\\
    &=(-1)^sD(\alpha)\qquad s={{n}\choose {2}}
\end{align*}
\end{proof}
\begin{proposition}
If $\{\alpha_1,\cdots, \alpha_n\}$ is a basis for $K$ over $\Q$, where $K$ is a number field, then:
$$D(\alpha_1,\cdots, \alpha_n)=\text{det}[(Tr_K(\alpha_i\alpha_j))_{ij}]\in M_n(\Q)$$
\end{proposition}
\begin{proof}
$$Tr_K(\alpha)=\sum_{k=1}^n\sigma_k(\alpha)$$
Thus,
$$Tr(\alpha_i\alpha_j)=\sum_{k=1}^n\sigma_k(\alpha_i\alpha_j)=\sum_{k=1}^n\sigma_k(\alpha_i)\sigma_k(\alpha_j)$$
Then, constructing our matrix of traces, we get that:
$$(Tr(\alpha_i\alpha_j))_{ij}=((\sigma_j(\alpha_i))_{ij}(\sigma_i(\alpha_j)_{ij}))_{ij}$$
$$\text{det}((Tr(\alpha_i\alpha_j)_{ij}))= \text{det}(\sigma_i(\alpha_j)_{ij})^2 = D(\alpha_1,\cdots, \alpha_n)$$
\end{proof}
\newpage
\subsection{Quadratic Fields}
\begin{definition}
A \underline{quadratic field} is an algebraic number field $K$ of degree 2 over $\Q$.
\end{definition}
\begin{proposition}
All quadratic fields are of the form $\Q(\sqrt{d})$ for some square-free $d\in\Z$. 
\end{proposition}
\begin{proof}
Let $K=\Q(\alpha)$. for some $\alpha\in\mathcal{O}$. Since $K$ is a quadratic field, the minimum polynomial of $\alpha$ is of the form $p(x)=x^2+ax+b$, where $a,b\in\Z$. This gives us that:
$$\alpha = \frac{-a\pm \sqrt{a^2-4b}}{2}$$
Now, we write: $a^2-4b= r^2d$, where $d$ is square free. Then, 
$$\alpha = \frac{-a\pm r\sqrt{d}}{2}$$
Thus, $\Q(\alpha)=\Q\left(\frac{-a\pm r\sqrt{d}}{2}\right)=\Q(\sqrt{d})$.
\end{proof}
\begin{remark}
If $d<0$, $\Q(\sqrt{d})$ is called an imaginary quadratic field. If $d>0$, this is instead called a real quadratic field. 
\end{remark}
\end{document}
