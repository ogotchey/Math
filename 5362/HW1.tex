\documentclass{article}
\usepackage{amsmath}
\usepackage{amssymb}
\usepackage{amsthm}
\usepackage{mathrsfs}

\newtheorem{theorem}{Theorem}[section]
\newtheorem{lemma}[theorem]{Lemma}
\newtheorem{proposition}[theorem]{Proposition}
\newtheorem{corollary}[theorem]{Corollary}
\newtheorem{definition}[theorem]{Definition}
\newtheorem{construction}[theorem]{Construction}
\newtheorem{example}[theorem]{Example}
\newtheorem{notation}[theorem]{Notation}
\newtheorem{remark}[theorem]{Remark}
\newtheorem{chunk}[theorem]{}

\DeclareMathOperator{\Z}{\mathbb{Z}}
\DeclareMathOperator{\N}{\mathbb{N}}
\DeclareMathOperator{\C}{\mathbb{C}}
\DeclareMathOperator{\scrO}{\mathscr{O}}

\begin{document}
\title{MATH 5362 Homework}
\author{Orin Gotchey}
\maketitle
\section{Homework 1}
\begin{lemma}
If $\alpha^n\in\scrO_{K}$ for some natural number $n\in\N$ and field $K$, then $\alpha\in\scrO_{K}$
\end{lemma}
\begin{proof}
Let $\alpha,\;n\;,K$ be as stated.  Then there exists some $g(x)\in\Z[x]$, written
\begin{equation}
g(x) = \Sigma_{i=0}^{deg(g)}(\beta_ix^i)
\end{equation}
where $\beta_i\in\Z$, and such that $g(\alpha^n)=0$.  But $\alpha$ must be a root of:
\begin{equation}
h(x) := g(x^n) \in\Z[x]
\end{equation}
\end{proof}
This in fact makes $\scrO_K$ into a radical $K$-ideal
\begin{proposition}{Problem 1}

$\theta := \frac{10^{\frac{2}{3}}-1}{\sqrt{-3}}$ is an algebraic integer.
\end{proposition}
\begin{proof}
\begin{equation}
\omega := (-3\cdot\theta^2)+1 = 10^{\frac{4}{3}} - 2*10^{\frac{2}{3}}
\end{equation}
Since $\scrO$ is a ring, and since $10^{\frac{2}{3}}$ is a root of $f(x) = x^3-100$, and is therefore an algebraic integer, we get: $\omega\in\scrO$.  That is, there exists some $f\in\Z[x]$ such that $f(\omega)=0$.  But then $f(-3\cdot\theta^2)=0$ gives rise to another $g(x)\in\Z[x]$ such that $g(\theta)=0$.  So $\theta\in\scrO$, as required;
\end{proof}
\begin{proposition}{Problem 2}\\
For any $n\in\N$, the complex value: $\alpha = \frac{\sqrt{m}+1}{\sqrt{2}}\in\scrO$
\end{proposition}
\end{document}